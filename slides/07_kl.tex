\documentclass[lualatex,handout]{beamer}
\setbeamertemplate{footline}[frame number]
\usepackage{luatexja}
\usepackage{amsmath,amssymb}

\usepackage{thm-restate}

%\usetheme{Berlin}
\usecolortheme{rose}

\usepackage{tikz}
\usepackage{pgfplots}
\pgfplotsset{compat=1.18}

%\usepackage[haranoaji]{luatexja-preset}
\usepackage[deluxe,ipaex]{luatexja-preset}
\renewcommand{\kanjifamilydefault}{\gtdefault}
%\setmainjfont{HaranoAjiGothic-Regular}

\usepackage{unicode-math}
%\setmathfont{Fira Math}
\setmathfont{STIX Two Math}
\setmathfont{STIX Two Math}[range=bfup/{Latin,latin,num,Greek,greek}]
\setmathfont{STIX Two Math}[range=bfit/{Latin,latin}]
\setmathfont{STIX Two Math}[range={"0000-"FFFF}]
\setmathrm{STIX Two Math}[StylisticSet=8]

%\usefonttheme{professionalfonts}

\usepackage{luacolor}

\newcommand{\mycolor}[2]{%
  \begingroup
  \colorlet{currentcolor}{.}%
  \color{#1}#2%
  \color{currentcolor}%
  \endgroup
}
\newcommand{\emm}[1]{\mycolor{red}{#1}}
\newcommand{\expt}[1]{\mathbb{E}\left[#1\right]}
\newcommand{\var}[1]{\mathbb{V}\left[#1\right]}
\newcommand{\cov}[1]{\mathsf{Cov}\left[#1\right]}
\newcommand{\vc}[1]{\mathsf{Var}\left[#1\right]}


\usepackage{xspace}
%\usepackage{bm}
%\newcommand\bm[1]{{\mathbf{#1}}}
\newcommand\defiff{\stackrel{\text{def}}{\iff}}
\newcommand\dx{{\,\mathrm{d}x}}
\newcommand\KL[2]{D\left(#1\,\|\,#2\right)}

\theoremstyle{definition}

\title{確率・統計基礎: KLダイバージェンスの性質}
\author{森 立平}
\date{}



\begin{document}
\begin{frame}[plain]
\maketitle
\end{frame}


\begin{frame}{Kullback--Leibler divergence}
\begin{definition}[Kullback--Leibler divergence]
$D\colon \mathcal{P}\times\mathcal{P}\to\mathbb{R}$を以下で定義する
\begin{align*}
\KL{q}{p} &\coloneq \sum_{k=1}^m q_k\log \frac{q_k}{p_k}.
\end{align*}
ただし、$0\log 0 = 0\log\frac{0}{0}=0$とし、$q\log\frac{q}{0}$は$q>0$について$+\infty$とする。
%ただし、ある$k$について$p_k=0,\,q_k>0$のとき$\KL{q}{p}$は定義されない。
\end{definition}
\end{frame}

\begin{frame}{サノフの定理}
%\begin{theorem}[サノフの定理]
\begin{restatable}[サノフの定理]{theorem}{sanov}
\footnotesize
%\begin{align*}
%-\inf_{q \in\Gamma^\circ} \KL{q}{p}
%&\le \liminf_{N\to\infty}\frac1N\log \Pr\left(P_{\symbf{X}} \in \Gamma\right)\\
%\limsup_{N\to\infty}\frac1N\log \Pr\left(P_{\symbf{X}} \in \mathcal{D}\right)
%&\le-\inf_{q \in\mathcal{D}^\circ} \KL{q}{p}
%\end{align*}
任意の$\Gamma\subseteq\mathcal{P}$について
%と$N\in\mathbb{Z}_{>0}$について
%\begin{align*}
%\Pr\left(P_{\symbf{X}} \in \mathcal{D}\right) &\le (N+1)^m \mathrm{e}^{-N\inf_{q\in \mathcal{D}} \emm{\KL{q}{p}}}.
%\end{align*}
%また任意の開集合$\mathcal{D}\subseteq\mathcal{P}$について
\begin{align*}
\frac1N\log\Pr\left(\widehat{P}_{\symbf{X}} \in \Gamma\right) &\le -\inf_{q\in \Gamma} \emm{\KL{q}{p}} + \frac{m\log(N+1)}N\quad\text{for any $N\in\mathbb{Z}_{>0}$}\\
\liminf_{N\to\infty}\frac1N\log \Pr\left(\widehat{P}_{\symbf{X}} \in \Gamma\right) &\ge -\inf_{q\in \Gamma^\circ} \emm{\KL{q}{p}}.
\end{align*}
ただし、$\Gamma^\circ$は$\Gamma$の内点の集合とする。
\end{restatable}
%\end{theorem}
\begin{corollary}
\footnotesize
$\Gamma\subseteq\mathcal{P}$について$\emm{\Gamma\subseteq\overline{\Gamma^\circ}}$のとき、
\begin{align*}
\lim_{N\to\infty}\frac1N\log \Pr\left(\widehat{P}_{\symbf{X}} \in \Gamma\right) &= -\min_{q\in \overline{\Gamma}} \emm{\KL{q}{p}}.
\end{align*}
ただし、$\overline{\Gamma}$は$\Gamma$の閉包とする。
\end{corollary}
\end{frame}

\begin{frame}{KL divergenceの性質}
\begin{enumerate}
\setlength{\itemsep}{2em}
\item $\KL{q}{p}\ge 0$ で等号は$q=p$のときのみ成り立つ。
\item 凸関数である。
$\KL{\lambda q+(1-\lambda)q'}{\lambda p+(1-\lambda)p'}
\le
\lambda\KL{q}{p} + (1-\lambda)\KL{q'}{p'}$.
%\item 
\end{enumerate}
\end{frame}

\begin{frame}{イェンセンの不等式}
\footnotesize
\begin{lemma}[\small イェンセンの不等式]
凸関数$f\colon\mathbb{R}\to\mathbb{R}$と確率変数$X\colon\Omega\to\mathbb{R}$について、
\begin{align*}
\expt{f(X)} &\ge f(\expt{X}).
\end{align*}
また、$f$が狭義凸のとき等号は$X$が決定的($\Pr(X=\expt{X})=1$)な場合にのみ成り立つ。
\end{lemma}
\begin{proof}
$f(x)$の$x=\expt{X}$における``接線''を
\begin{align*}
y&=a(x-\expt{X}) + f(\expt{X})
\end{align*}
とおくと、$f$の凸性より接線は$f$より下側にある。
つまり
\begin{align*}
f(x) &\ge a(x-\expt{X}) + f(\expt{X})\qquad \forall x\in\mathbb{R}
\end{align*}
である。$x=X$として期待値を取ることで目的の不等式を得る。

$f$が狭義凸の場合、$f$と接線は一点のみを共有するので決定的でない場合は狭義の不等式が成り立つ。
\end{proof}
\end{frame}


\begin{frame}{Log sum 不等式}
\footnotesize
\begin{lemma}[Log sum 不等式]
任意の$a_1,\dotsc,a_m,\,b_1,\dotsc,\,b_m\in\mathbb{R}_{>0}$について
\begin{align*}
\sum_k a_k\log\frac{a_k}{b_k}&\ge
\left(\sum_k a_k\right)\log\frac{\left(\sum_ka_k\right)}{\left(\sum_kb_k\right)}.
\end{align*}
等号は$\frac{a_k}{b_k}$が一定の場合のみ成り立つ。
\end{lemma}
\begin{proof}
\vspace{-1em}
\begin{align*}
\sum_k a_k\log\frac{a_k}{b_k}
%&=\sum_k \emm{b_k}\frac{a_k}{\emm{b_k}}\log\frac{a_k}{b_k}\\
&=\emm{\left(\sum_k b_k\right)}\sum_k \frac{\emm{b_k}}{\emm{\sum_{k'} b_{k'}}}\frac{a_k}{\emm{b_k}}\log\frac{a_k}{b_k}\\
&\ge\left(\sum_k b_k\right)\left(\sum_k \frac{b_k}{\sum_{k'} b_{k'}}\frac{a_k}{b_k}\right)
\log\left(\sum_k \frac{b_k}{\sum_{k'} b_{k'}}\frac{a_k}{b_k}\right)\\
&\hspace{10em} \text{($x\log x$の凸性とイェンセンの不等式)}\\
&=\left(\sum_k a_k\right)\log\frac{\left(\sum_ka_k\right)}{\left(\sum_kb_k\right)}.\qedhere
\end{align*}
\end{proof}
零を含む場合も$\to 0$極限を考えることで不等式が示せる。
\end{frame}

\if0
\begin{frame}{KL-divergenceの非負性}
\begin{lemma}
%\begin{itemize}
任意の$p,\,q\in\mathcal{P}$について$\KL{q}{p}\ge 0$.
%\item $\KL{q}{p}\ge 0$
%\item $\KL{q}{p}= 0\iff q=p$
%\end{itemize}
\end{lemma}
\begin{proof}
\begin{align*}
\KL{q}{p} &= \sum_{k=1}^m q_k\log \frac{q_k}{p_k}\\
&= -\sum_{k=1}^m q_k\log \frac{p_k}{q_k}\\
&\ge -\log \sum_{k=1}^m q_k\frac{p_k}{q_k}\qquad\text{(イェンセンの不等式)}\\
&=0.\qedhere
\end{align*}
%$q=p$のとき、$\KL{q}{p}=0$は自明。
\end{proof}
\end{frame}
\fi

\begin{frame}{KL-divergenceの凸性}
\small
\begin{lemma}
任意の$p,p',q,q'\in\mathcal{P}$と$\lambda\in[0,1]$について
\begin{align*}
\KL{\lambda q+(1-\lambda)q'}{\lambda p+(1-\lambda)p'}
&\le
\lambda\KL{q}{p} + (1-\lambda)\KL{q'}{p'}.
\end{align*}
\end{lemma}
\begin{proof}
\vspace{-2em}
\begin{align*}
%&\KL{\lambda q+(1-\lambda)q'}{\lambda p+(1-\lambda)p'}\\
%&= \sum_k (\lambda q_k+(1-\lambda)q'_k)\log
%\frac{\lambda q_k+(1-\lambda)q'_k}{\lambda p_k+(1-\lambda)p'_k}\\
%&= \sum_k \emm{(\lambda p_k+(1-\lambda)p'_k)}
%\frac{\lambda q_k+(1-\lambda)q'_k}{\emm{\lambda p_k+(1-\lambda)p'_k}}
%\log
%\frac{\lambda q_k+(1-\lambda)q'_k}{\lambda p_k+(1-\lambda)p'_k}\\
&(\lambda q_k+(1-\lambda)q'_k)\log \frac{\lambda q_k+(1-\lambda)q'_k}{\lambda p_k+(1-\lambda)p'_k}\\
&\le \lambda q_k\log \frac{\lambda q_k}{\lambda p_k}
+ (1-\lambda) q_k\log \frac{(1-\lambda) q_k}{(1-\lambda) p_k}\qquad\text{(Log sum不等式)}\\
&= \lambda q_k\log \frac{q_k}{p_k}
+ (1-\lambda) q_k\log \frac{q_k}{p_k}.
\end{align*}
両辺を$k$について和を取ると目的の不等式を得る。
\end{proof}
$p'=p$や$q'=q$の場合を考えると、
\begin{align*}
\KL{\lambda q+(1-\lambda)q'}{p} &\le \lambda\KL{q}{p} + (1-\lambda)\KL{q'}{p}\\
\KL{q}{\lambda p+(1-\lambda)p'} &\le \lambda\KL{q}{p} + (1-\lambda)\KL{q}{p'}.
\end{align*}
\end{frame}

\if0
\begin{frame}{ラグランジュの未定乗数法}
\end{frame}

\begin{frame}{最大エントロピー原理}
\footnotesize
\begin{theorem}
$\Gamma=\{q\in\mathcal{P}\mid Y\sim\mathrm{Cat}(q),\,\expt{f_s(Y)}\ge a_s,\, s\in\{1,\dotsc,t\}\}$について
\begin{align*}
\min_{q}\colon& \KL{q}{p}\\
\text{subject to}\colon& \sum_{k=1}^m q_k = 1,\qquad
q_k\ge 0\quad\forall k\in\{1,2,\dotsc,m\}\\
&\sum_{k=1}^m q_k f_s(k) \ge a\qquad \forall s\in\{1,2,\dotsc,t\} 
\end{align*}
の解は$(\lambda_s\in\mathbb{R})_{s=1,\dotsc,t}$について
\begin{align*}
q_k &= \frac1{M_{f(X)}(\lambda)} p_k\mathrm{e}^{\sum_{s=1}^t \lambda_s f_s(k)}\qquad\text{(Gibbs--Boltzmann分布)}\\
M_{f(X)}(\lambda) &\coloneq \sum_{k=1}^m p_k\mathrm{e}^{\sum_{s=1}^t \lambda_s f_s(k)} = \expt{\mathrm{e}^{\langle \lambda,\, f(X)\rangle}} \qquad \text{(分配関数)}
\end{align*}
と表すことができ、目的関数の値は
\begin{align*}
\KL{q}{p} &= \sum_{s}\lambda_s\expt{f_s(Y)} -\log M_{f(X)}(\lambda)
\end{align*}
\end{theorem}
\end{frame}

\begin{frame}{ラグランジュの未定乗数法}
\begin{align*}
\mathcal{L}(q;\lambda,\,\rho,\,\nu) &= \sum_{k=1}^m q_k\log\frac{q_k}{p_k}\\
&\quad - \sum_{s=1}^t \lambda_s \left(\sum_{k=1}^m q_k f_s(k) - a_s\right) - \sum_{k=1}^m \rho_k q_k + \nu \left(\sum_{k=1}^m q_k - 1\right)
\end{align*}
\begin{align*}
\inf_{q}\sup_{\lambda\ge 0,\, \rho\ge0,\, \nu} \mathcal{L}(q;\lambda,\rho,\nu) &= 
\sup_{\lambda\ge 0,\, \rho\ge0,\, \nu}\inf_{q} \mathcal{L}(q;\lambda,\rho,\nu)
\end{align*}
\begin{align*}
\frac{\partial \mathcal{L}}{\partial q_k} &= \log\frac{q_k}{p_k} + 1 - \sum_{s=1}^t \lambda_s f_s(k) - \rho_k + \nu = 0\\
\iff \log q_k &= \log p_k - 1 + \sum_{s=1}^t \lambda_s f_s(k) + \rho_k - \nu\\
\iff q_k &= p_k \mathrm{e}^{\sum_{s=1}^t \lambda_s f_s(k) - 1 - \nu}\\
\end{align*}
\end{frame}

\begin{frame}{サノフの定理を用いたクラメールの定理の導出}
\footnotesize
集合$A$が有限集合で$\mathbb{R}$の部分集合のとき、
$\Gamma\coloneq\{q\in \mathcal{P}\mid Y\sim\mathrm{Cat}(q),\,\expt{Y}\in C\}$とすると
\begin{align*}
\Pr\left(\frac1N(X_1+\dotsb+X_N)\in C\right) &= \Pr(P_{\symbf{X}}\in\Gamma)
\end{align*}
任意の$q\in\mathcal{P}$と$t\in\mathbb{R}$について
\begin{align*}
K_X(t) &= \log\expt{\mathrm{e}^{tX}}\\
&=\log\sum_k p_k \mathrm{e}^{ta_k}\\
&=\log\sum_k \emm{q_k}\frac{p_k}{\emm{q_k}} \mathrm{e}^{ta_k}\\
&\ge \sum_k q_k\log\frac{p_k}{q_k} \mathrm{e}^{ta_k}\qquad\text{($\log$の凹性とイェンセンの不等式)}\\
&= -\KL{q}{p} + t \expt{Y}
\end{align*}
よって
\begin{align*}
\KL{q}{p}&\ge t\expt{Y} - K_X(t)\qquad\forall q\in\mathcal{P},\,t\in\mathbb{R}\\
\inf_{q\in\Gamma}\KL{q}{p}&\ge \inf_{a\in C}\{ta - K_X(t)\}\qquad\forall t\in \mathbb{R}
\end{align*}
\begin{align*}
\inf_{q\in\Gamma}\KL{q}{p}&\ge \inf_{a\in C}\sup_{t\in \mathbb{R}}\left\{ta - K_X(t)\right\}
=\inf_{a\in C} I_X(a)
\end{align*}

\end{frame}
\fi

\begin{frame}{エントロピー}
\begin{definition}[エントロピー]
$H\colon \mathcal{P}\to\mathbb{R}$を以下で定義する
\begin{align*}
H(p) &\coloneq -\sum_{k=1}^m p_k\log p_k.
\end{align*}
ただし、$0\log 0 = 0$とする。
$\log$の底は$m$とすることも多い。
%ただし、ある$k$について$p_k=0,\,q_k>0$のとき$\KL{q}{p}$は定義されない。
\end{definition}
$q$を$\{1,2,\dotsc,m\}$上の一様分布とすると、
$H(p) = \KL{p}{q} - \log m$である。
\end{frame}

\begin{frame}{多項係数とエントロピー}
\begin{lemma}
$N,N_1,\dotsc,N_m\in\mathbb{Z}_{\ge 0}$で$N_1+\dotsb+N_m=N$のとき、
\begin{align*}
\frac1{(N+1)^m}\mathrm{e}^{NH(p)}\le
\binom{N}{N_1\,N_2\,\dotsc,N_m} \le \mathrm{e}^{NH(p)}
\end{align*}
ただし$k=1,2,\dotsc,m$について$p_k\coloneq N_k/N$とする。
\end{lemma}
\begin{proof}
\begin{align*}
\frac1{(N+1)^m}&\le
\binom{N}{N_1\,N_2\,\dotsc,N_m} \prod_{k=1}^m (p_k)^{Np_k}\le 1\\
\iff\frac1{(N+1)^m}&\le
\binom{N}{N_1\,N_2\,\dotsc,N_m} \mathrm{e}^{-NH(p)}\le 1.
\end{align*}
\end{proof}
\end{frame}

\begin{frame}{データ圧縮とエントロピー}
データ圧縮の概要

\vspace{2em}
\begin{itemize}
\setlength{\itemsep}{2em}
\item データ$\symbf{x}\in A^N$がi.i.dでそれぞれが確率分布$p\in\mathcal{P}$に従っていると仮定する。
\item サノフの定理よりデータの経験分布$\widehat{P}_{\symbf{x}}$は高い確率で真の分布$p$に近い。
\item よってあり得る$\symbf{x}$の数は大体$\binom{N}{p_1N\,p_2N\,\dotsm\,p_mN}\approx \mathrm{e}^{NH(p)}$である。
\item よって長さ $\log_m \mathrm{e}^{N H(p)} = N H(p)/(\log m)$ の$A$の列に圧縮できる。ここで $H(p)/(\log m)$はエントロピーの定義の$\log$の底を$m$にしたものに等しい。
\end{itemize}
\end{frame}

\begin{frame}{確率論振り返り}
\begin{itemize}
\setlength{\itemsep}{2em}
\item 確率空間$(\Omega, P)$. 完全加法性。
\item 大数の法則、クラメールの定理。
\item 中心極限定理
\item サノフの定理
\end{itemize}
\end{frame}

\begin{frame}{課題}
\begin{itemize}
\item 二元エントロピー関数$h(p)\coloneq -p\log_2 p - (1-p)\log_2(1-p)\quad\text{for } p\in[0,1]$のグラフを描け。凸性に気をつけること。
%\item 確率変数$X\sim\mathrm{Multinom}(p)$について
%\begin{align*}
%\lim_{N\to\infty} \frac1N\log\Pr\left(\KL{P_{\symbf{X}}}{p}\ge a\right)
%\end{align*}
%をもとめよ。
%\item $\Gamma_a\coloneq\{q\in\mathcal{P}\mid \sum_k q_k k \ge a\}$
\end{itemize}
\end{frame}

\end{document}
