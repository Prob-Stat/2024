\documentclass[lualatex,handout]{beamer}
\setbeamertemplate{footline}[frame number]
\usepackage{luatexja}
\usepackage{amsmath,amssymb}

\usepackage{thm-restate}

%\usetheme{Berlin}
\usecolortheme{rose}

\usepackage{tikz}
\usepackage{pgfplots}
\pgfplotsset{compat=1.18}

%\usepackage[haranoaji]{luatexja-preset}
\usepackage[deluxe,ipaex]{luatexja-preset}
\renewcommand{\kanjifamilydefault}{\gtdefault}
%\setmainjfont{HaranoAjiGothic-Regular}

\usepackage{unicode-math}
%\setmathfont{Fira Math}
\setmathfont{STIX Two Math}
\setmathfont{STIX Two Math}[range=bfup/{Latin,latin,num,Greek,greek}]
\setmathfont{STIX Two Math}[range=bfit/{Latin,latin}]
\setmathfont{STIX Two Math}[range={"0000-"FFFF}]
\setmathrm{STIX Two Math}[StylisticSet=8]

%\usefonttheme{professionalfonts}

\usepackage{luacolor}

\newcommand{\mycolor}[2]{%
  \begingroup
  \colorlet{currentcolor}{.}%
  \color{#1}#2%
  \color{currentcolor}%
  \endgroup
}
\newcommand{\emm}[1]{\mycolor{red}{#1}}
\newcommand{\expt}[1]{\mathbb{E}\left[#1\right]}
\newcommand{\var}[1]{\mathbb{V}\left[#1\right]}
\newcommand{\cov}[1]{\mathsf{Cov}\left[#1\right]}
\newcommand{\vc}[1]{\mathsf{Var}\left[#1\right]}


\usepackage{xspace}
%\usepackage{bm}
%\newcommand\bm[1]{{\mathbf{#1}}}
\newcommand\defiff{\stackrel{\text{def}}{\iff}}
\newcommand\dx{{\,\mathrm{d}x}}
\newcommand\KL[2]{D\left(#1\,\|\,#2\right)}

\theoremstyle{definition}

\title{確率・統計基礎: KLダイバージェンスの性質}
\author{森 立平}
\date{}



\begin{document}
\begin{frame}[plain]
\maketitle
\end{frame}


\begin{frame}{Kullback--Leibler divergence}
\begin{definition}[Kullback--Leibler divergence]
$D\colon \mathcal{P}\times\mathcal{P}\to\mathbb{R}$を以下で定義する
\begin{align*}
\KL{q}{p} &\coloneq \sum_{k=1}^m q_k\log \frac{q_k}{p_k}.
\end{align*}
ただし、$0\log 0 = 0\log\frac{0}{0}=0$とし、$q\log\frac{q}{0}$は$q>0$について$+\infty$とする。
%ただし、ある$k$について$p_k=0,\,q_k>0$のとき$\KL{q}{p}$は定義されない。
\end{definition}
\end{frame}


\begin{frame}{KL-divergenceの性質}
\begin{lemma}
%\begin{itemize}
任意の$p,\,q\in\mathcal{P}$について$\KL{q}{p}\ge 0$.
%\item $\KL{q}{p}\ge 0$
%\item $\KL{q}{p}= 0\iff q=p$
%\end{itemize}
\end{lemma}
\begin{proof}
\begin{align*}
\KL{q}{p} &= \sum_{k=1}^m q_k\log \frac{q_k}{p_k}\\
&= -\sum_{k=1}^m q_k\log \frac{p_k}{q_k}\\
&\ge -\log \sum_{k=1}^m q_k\frac{p_k}{q_k}\qquad\text{(イェンセンの不等式)}\\
&=0.\qedhere
\end{align*}
%$q=p$のとき、$\KL{q}{p}=0$は自明。
\end{proof}
\end{frame}

\begin{frame}{KL-divergenceの凸性}
a
\end{frame}

\begin{frame}{サノフの定理}
%\begin{theorem}[サノフの定理]
\begin{restatable}[サノフの定理]{theorem}{sanov}
\footnotesize
%\begin{align*}
%-\inf_{q \in\Gamma^\circ} \KL{q}{p}
%&\le \liminf_{N\to\infty}\frac1N\log \Pr\left(P_{\symbf{X}} \in \Gamma\right)\\
%\limsup_{N\to\infty}\frac1N\log \Pr\left(P_{\symbf{X}} \in \mathcal{D}\right)
%&\le-\inf_{q \in\mathcal{D}^\circ} \KL{q}{p}
%\end{align*}
任意の$\Gamma\subseteq\mathcal{P}$について
%と$N\in\mathbb{Z}_{>0}$について
%\begin{align*}
%\Pr\left(P_{\symbf{X}} \in \mathcal{D}\right) &\le (N+1)^m \mathrm{e}^{-N\inf_{q\in \mathcal{D}} \emm{\KL{q}{p}}}.
%\end{align*}
%また任意の開集合$\mathcal{D}\subseteq\mathcal{P}$について
\begin{align*}
\frac1N\log\Pr\left(\widehat{P}_{\symbf{X}} \in \Gamma\right) &\le -\inf_{q\in \Gamma} \emm{\KL{q}{p}} + \frac{m\log(N+1)}N\quad\text{for any $N\in\mathbb{Z}_{>0}$}\\
\liminf_{N\to\infty}\frac1N\log \Pr\left(\widehat{P}_{\symbf{X}} \in \Gamma\right) &\ge -\inf_{q\in \Gamma^\circ} \emm{\KL{q}{p}}.
\end{align*}
ただし、$\Gamma^\circ$は$\Gamma$の内点の集合とする。
\end{restatable}
%\end{theorem}
\begin{corollary}
\footnotesize
$\Gamma\subseteq\mathcal{P}$について$\emm{\Gamma\subseteq\overline{\Gamma^\circ}}$のとき、
\begin{align*}
\lim_{N\to\infty}\frac1N\log \Pr\left(\widehat{P}_{\symbf{X}} \in \Gamma\right) &= -\min_{q\in \overline{\Gamma}} \emm{\KL{q}{p}}.
\end{align*}
\end{corollary}
\end{frame}

\begin{frame}{ラグランジュの未定乗数法}
\end{frame}

\begin{frame}{最大エントロピー原理}
\begin{theorem}
$\Gamma=\{q\in\mathcal{P}\mid \expt{f_s(Y)}\ge a_s,\, s\in\{1,\dotsc,t\}\}$について
\begin{align*}
\min_{q}\colon& \KL{q}{p}\\
\text{subject to}\colon& \sum_{k=1}^m q_k = 1\\
&q_k\ge 0\\
&\sum_{k=1}^m q_k f_s(k) \ge a\qquad \text{for } s\in\{1,2,\dotsc,t\} 
\end{align*}
の解は$(\lambda_s\in\mathbb{R})_{s=1,\dotsc,t}$
\begin{align*}
q_k &= \frac1{Z(\lambda)} p_k\mathrm{e}^{\sum_{s=1}^t \lambda_s f_s(k)}\\
Z(\lambda) &\coloneq \sum_{k=1}^m p_k\mathrm{e}^{\sum_{s=1}^t \lambda_s f_s(k)} \qquad \text{(分配関数)}
\end{align*}
\end{theorem}
\end{frame}

\begin{frame}{ラグランジュの未定乗数法}
\begin{align*}
\mathcal{L}(q;\lambda,\,\rho,\,\nu) &= \sum_{k=1}^m q_k\log\frac{q_k}{p_k}\\
&\quad - \sum_{s=1}^t \lambda_s \left(\sum_{k=1}^m q_k f_s(k) - a_s\right) - \sum_{k=1}^m \rho_k q_k + \nu \left(\sum_{k=1}^m q_k - 1\right)
\end{align*}
\begin{align*}
\inf_{q}\sup_{\lambda\ge 0,\, \rho\ge0,\, \nu} \mathcal{L}(q;\lambda,\rho,\nu) &= 
\sup_{\lambda\ge 0,\, \rho\ge0,\, \nu}\inf_{q} \mathcal{L}(q;\lambda,\rho,\nu)
\end{align*}
\begin{align*}
\frac{\partial \mathcal{L}}{\partial q_k} &= \log\frac{q_k}{p_k} + 1 - \sum_{s=1}^t \lambda_s f_s(k) - \rho_k + \nu = 0\\
\iff \log q_k &= \log p_k - 1 + \sum_{s=1}^t \lambda_s f_s(k) + \rho_k - \nu\\
\iff q_k &= p_k \mathrm{e}^{\sum_{s=1}^t \lambda_s f_s(k) - 1 - \nu}\\
\end{align*}
\end{frame}

\begin{frame}{確率論振り返り}
\begin{itemize}
\setlength{\itemsep}{2em}
\item 確率空間$(\Omega, P)$. 完全加法性。
\item 大数の法則、クラメールの定理。
\item 中心極限定理
\item サノフの定理
\end{itemize}
\end{frame}

\end{document}
