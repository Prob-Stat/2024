\documentclass[lualatex,handout]{beamer}
\setbeamertemplate{footline}[frame number]
\usepackage{luatexja}
\usepackage{amsmath,amssymb}

\usepackage{thm-restate}

%\usetheme{Berlin}
\usecolortheme{rose}

\usepackage{tikz}
\usepackage{pgfplots}
\pgfplotsset{compat=1.18}

%\usepackage[haranoaji]{luatexja-preset}
\usepackage[deluxe,ipaex]{luatexja-preset}
\renewcommand{\kanjifamilydefault}{\gtdefault}
%\setmainjfont{HaranoAjiGothic-Regular}

\usepackage{unicode-math}
%\setmathfont{Fira Math}
\setmathfont{STIX Two Math}
\setmathfont{STIX Two Math}[range=bfup/{Latin,latin,num,Greek,greek}]
\setmathfont{STIX Two Math}[range=bfit/{Latin,latin}]
\setmathfont{STIX Two Math}[range={"0000-"FFFF}]
\setmathrm{STIX Two Math}[StylisticSet=8]

%\usefonttheme{professionalfonts}

\usepackage{luacolor}

\newcommand{\mycolor}[2]{%
  \begingroup
  \colorlet{currentcolor}{.}%
  \color{#1}#2%
  \color{currentcolor}%
  \endgroup
}
\newcommand{\emm}[1]{\mycolor{red}{#1}}
\newcommand{\expt}[1]{\mathbb{E}\left[#1\right]}
\newcommand{\var}[1]{\mathbb{V}\left[#1\right]}
\newcommand{\cov}[1]{\mathsf{Cov}\left[#1\right]}
\newcommand{\vc}[1]{\mathsf{Var}\left[#1\right]}

\DeclareMathOperator*{\argmax}{arg\,max}


\usepackage{xspace}
%\usepackage{bm}
%\newcommand\bm[1]{{\mathbf{#1}}}
\newcommand\defiff{\stackrel{\text{def}}{\iff}}
\newcommand\dx{{\,\mathrm{d}x}}
\newcommand\KL[2]{D\left(#1\,\|\,#2\right)}
\newcommand\dtv{d_{\mathrm{TV}}}

\theoremstyle{definition}

\title{確率・統計基礎: 母数推定}
\author{森 立平}
\date{}



\begin{document}
\begin{frame}[plain]
\maketitle
\end{frame}



\begin{frame}{連続確率変数の場合}
\begin{itemize}
\setlength{\itemsep}{2em}
\item $x\in A$: データ、知っているもの、$A\subseteq \mathbb{R}$とか$A\subseteq\mathbb{R}^n$とか
%\item $\emm{\theta \in \Theta}\colon$ パラメータ、母数、知りたいもの。
%$\Theta\subseteq\mathbb{R}$とか$\Theta\subseteq\mathbb{R}^n$とか。
%\item $s\in B$: パラメータ、推定したいもの(今日は離散の場合を考える。連続のときは$\theta$という記号を使うことが多い)
\item $\theta\in \Theta$: パラメータ、\emm{母数}、知りたいもの、$\Theta\subseteq\mathbb{R}$とか$\Theta\subseteq\mathbb{R}^n$とか
\item $p(\theta)$: 事前確率密度関数
\item $p(x \mid \theta)$: 尤度(条件付き確率密度関数)
\item $p(\theta \mid x)\coloneq p(x\mid \theta)p(\theta) / p(x)$: 事後確率密度関数
\end{itemize}
\end{frame}

\begin{frame}{誤差関数}
\small
$\widehat{\theta}\colon A\to \Theta$
\begin{align*}
\expt{d(\widehat{\theta}(X), \theta)}
&=
\int_{A\times\Theta} d(\widehat{\theta}(x),\theta) p(x\mid\theta) p(\theta) \mathrm{d}x\mathrm{d}\theta
\end{align*}
\end{frame}

\begin{frame}{二乗誤差}
\small
\begin{align*}
\mathrm{MSE}(\theta)&= \expt{(\widehat{\theta}(X)-\theta)^2\mid\theta}\\
&= \expt{\left(\widehat{\theta}(X)-\expt{\widehat{\theta}(X)\mid\theta}+\expt{\widehat{\theta}(X)\mid\theta}-\theta\right)^2\mid\theta}\\
&= \expt{\left(\widehat{\theta}(X)-\expt{\widehat{\theta}(X)\mid\theta}\right)^2\mid\theta}+\expt{\left(\expt{\widehat{\theta}(X)\mid\theta}-\theta\right)^2\mid\theta}\\
&\quad + 2\expt{\left(\widehat{\theta}(X)-\expt{\widehat{\theta}(X)\mid\theta}\right)\left(\expt{\widehat{\theta}(X)\mid\theta}-\theta\right)\mid\theta}\\
&= \var{\widehat{\theta}(X)\mid\theta}+\left(\expt{\widehat{\theta}(X)\mid\theta}-\theta\right)^2\\
&= \var{\widehat{\theta}(X)\mid\theta}+\mathrm{Bias}(\widehat{\theta}\mid\theta)^2
%\expt{\widehat{\theta}(X)^2 - 2\widehat{\theta}(X)\theta + \theta^2}
\end{align*}
\begin{align*}
\mathrm{Bias}(\widehat{\theta}\mid\theta)&\coloneq\expt{\widehat{\theta}(X)\mid\theta}-\theta
\end{align*}
\begin{definition}[不偏推定量]
推定量$\widehat{\theta}\colon A\to\Theta$が以下を満たすとき、\emm{不偏推定量}であるという。
\begin{align*}
\mathrm{Bias}(\widehat{\theta}\mid\theta) &= 0 \qquad\forall \theta\in\Theta.
\end{align*}
\end{definition}
\end{frame}

\begin{frame}{推定量の平均}
\footnotesize
\begin{lemma}
$\widehat{\theta}_1,\,\widehat{\theta}_2,\dotsc,\widehat{\theta}_n\colon A\to\Theta$が不偏推定量のとき、
\begin{align*}
\widehat{\theta}^* &\coloneq \frac1n\sum_{k=1}^n\widehat{\theta}_k
\end{align*}
は不偏推定量で
\begin{align*}
\var{\widehat{\theta}^*(X)\mid\theta} &\le\frac1n\sum_{k=1}^n\var{\widehat{\theta}_k(X)\mid\theta}
\qquad\forall\theta\in\Theta.
\end{align*}
\end{lemma}
\begin{proof}
$\widehat{\theta}^*$が不偏推定量であることは期待値の線形性から明らか。
\begin{align*}
\var{\widehat{\theta}^*(X)\mid\theta} &=
\expt{\left(\frac1n\sum_{k=1}^n\widehat{\theta}_k(X) - \theta\right)^2\mid\theta}\\
&\le
\frac1n\sum_{k=1}^n\expt{\left(\widehat{\theta}_k(X) - \theta\right)^2\mid\theta}\quad\text{(イェンセンの不等式)}\qedhere
%\frac1{n^2} \left(\sum_{k=1}^n \var{\widehat{\theta}_k(X)\mid\theta} + 2\sum_{k<\ell} \cov{\widehat{\theta}_k,\,\widehat{\theta}_\ell}\right)
\end{align*}
\end{proof}
\end{frame}

\begin{frame}{対称な不偏推定量}
\small
\begin{align*}
p(\symbf{x}\mid \theta) &= \prod_{k=1}^N p(x_k\mid \theta).
\end{align*}
\begin{lemma}[対称な不偏推定量]
任意の不偏推定量$\widehat{\theta}\colon A^N\to\Theta$について、ある対称な不偏推定量$\widehat{\theta}^*$が存在して
\begin{align*}
\var{\widehat{\theta}^*(\symbf{X})\mid\theta} &\le \var{\widehat{\theta}(\symbf{X})\mid\theta}
\qquad\forall\theta\in\Theta.
\end{align*}
\end{lemma}
\begin{proof}
\begin{align*}
\widehat{\theta}^*(x_1,\dotsc,x_N) &\coloneq \frac1{N!} \sum_{\sigma\in S_N} \widehat{\theta}\left(x_{\sigma(1)},\dotsc,x_{\sigma(N)}\right)
\end{align*}
と定義すればよい。
\end{proof}
\end{frame}

\begin{frame}{加法的ノイズのモデル}
\footnotesize
\begin{align*}
X &= \theta + Z
\end{align*}
\begin{itemize}
\item $\expt{Z} = 0$.
\item $\var{Z} < \infty$.
\end{itemize}
\begin{theorem}
$\widehat{\theta}^*$が$Z$の分布によらず対称な不偏推定量となるとき、$\widehat{\theta}^*$は\emm{標本平均}
\begin{align*}
\widehat{\theta}^*(x_1,\dotsc,x_n)&\coloneq \frac1n \sum_{k=1}^n x_k
\end{align*}
である。
\end{theorem}
\begin{proof}
$n=1$のとき、
$\Pr(Z=0)=1$と仮定すると、
\begin{align*}
\theta&= \expt{\widehat{\theta}^*(X)\mid\theta}
= \widehat{\theta}^*(\theta)
\end{align*}
が任意の$\theta\in\Theta$について成り立つ。よって$\widehat{\theta}^*(x)=x$である。
\end{proof}
\end{frame}

\begin{frame}{加法的ノイズのモデル}
\small
ある$a\ne 0$と$p\in[0,1]$について、ノイズの分布を
\begin{align*}
\Pr(Z = a(p-1)) &= p&
\Pr(Z = ap) &= 1-p
\end{align*}
とすると$\expt{Z}=0$, $\var{Z}<\infty$である。

任意の$a\ne 0$と$p\in[0,1]$について
\begin{align*}
%\expt{\widehat{\theta}^*(\symbf{X})\mid\theta} &=
\theta&= \expt{\widehat{\theta}^*(X)\mid\theta}\\
 &= p \widehat{\theta}^*(\theta+a(p-1)) + (1-p) \widehat{\theta}^*(\theta+ap)\\
 &= p \widehat{\theta}^*(\theta+a(p-1)) + (1-p) \widehat{\theta}^*(\theta+ap)
\end{align*}
が成り立つ。
$a=2\theta,\,p=\frac12$のとき、
\begin{align*}
\theta &= \frac12 \widehat{\theta}^*(0) + \frac12 \widehat{\theta}^*(2\theta)
\end{align*}
よって$\widehat{\theta}^*(2\theta) = 2\theta - \widehat{\theta}^*(0)$が任意の$\theta\in\Theta$について成り立つ。
よって、ある$c\in\mathbb{R}$が存在して$\widehat{\theta}^*(x) = x - c$であるが、$\widehat{\theta}^*(\theta)=\theta$より$c=0$
\end{frame}

\begin{frame}{課題}
\small
$A=\{0,\,1\},\, B=\left\{p_0=\frac12,\,p_1=\frac13\right\}$とし、$k\in\{0,\,1\}$について
\begin{align*}
\Pr(X = 0 \mid S = p_k) &= 1-p_k&
\Pr(X = 1 \mid S = p_k) &= p_k
\end{align*}
とする。
このとき、 $\eta> 0,\,\kappa\in[0,1]$について、以下の尤度比検定関数$E(x)$で仮説検定することを考える。
\begin{align*}
E(x)&=
\begin{cases}
\vspace{.5em}
0&\text{if } \frac{\Pr(X = x\mid S = p_0)}{\Pr(X = x\mid S = p_1)} > \eta\\\vspace{.5em}
1&\text{if } \frac{\Pr(X = x\mid S = p_0)}{\Pr(X = x\mid S = p_1)} < \eta\\
\kappa&\text{otherwise.}
\end{cases}
\end{align*}
以下の問に答えよ。
\begin{enumerate}
\setlength{\itemsep}{1em}
\item $\eta=1$のとき$\alpha_E$, $\beta_E$をもとめよ。
\item 一般の$\eta>0$と$\kappa\in[0,1]$について、誤り確率$\alpha_E$, $\beta_E$をもとめよ。$\eta$の値で場合分けしてもとめること。
\item 任意の検定関数$E\colon A\to[0,1]$で実現可能な$(\alpha_E,\, \beta_E)$の範囲を図示せよ。
\end{enumerate}
\end{frame}


\end{document}
