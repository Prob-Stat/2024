\documentclass[lualatex,handout]{beamer}
\setbeamertemplate{footline}[frame number]
\usepackage{luatexja}
\usepackage{amsmath,amssymb}

\usepackage{thm-restate}

%\usetheme{Berlin}
\usecolortheme{rose}

\usepackage{tikz}
\usepackage{pgfplots}
\pgfplotsset{compat=1.18}

%\usepackage[haranoaji]{luatexja-preset}
\usepackage[deluxe,ipaex]{luatexja-preset}
\renewcommand{\kanjifamilydefault}{\gtdefault}
%\setmainjfont{HaranoAjiGothic-Regular}

\usepackage{unicode-math}
%\setmathfont{Fira Math}
\setmathfont{STIX Two Math}
\setmathfont{STIX Two Math}[range=bfup/{Latin,latin,num,Greek,greek}]
\setmathfont{STIX Two Math}[range=bfit/{Latin,latin}]
\setmathfont{STIX Two Math}[range={"0000-"FFFF}]
\setmathrm{STIX Two Math}[StylisticSet=8]

%\usefonttheme{professionalfonts}

\usepackage{luacolor}

\newcommand{\mycolor}[2]{%
  \begingroup
  \colorlet{currentcolor}{.}%
  \color{#1}#2%
  \color{currentcolor}%
  \endgroup
}
\newcommand{\emm}[1]{\mycolor{red}{#1}}
\newcommand{\expt}[1]{\mathbb{E}\left[#1\right]}
\newcommand{\var}[1]{\mathbb{V}\left[#1\right]}
\newcommand{\cov}[1]{\mathsf{Cov}\left[#1\right]}
\newcommand{\vc}[1]{\mathsf{Var}\left[#1\right]}

\DeclareMathOperator*{\argmax}{arg\,max}


\usepackage{xspace}
%\usepackage{bm}
%\newcommand\bm[1]{{\mathbf{#1}}}
\newcommand\defiff{\stackrel{\text{def}}{\iff}}
\newcommand\dx{{\,\mathrm{d}x}}
\newcommand\KL[2]{D\left(#1\,\|\,#2\right)}
\newcommand\dtv{d_{\mathrm{TV}}}

\theoremstyle{definition}

\title{確率・統計基礎: 母数推定}
\author{森 立平}
\date{}



\begin{document}
\begin{frame}[plain]
\maketitle
\end{frame}



\begin{frame}{連続確率変数の場合}
\begin{itemize}
\setlength{\itemsep}{2em}
\item $x\in A$: データ、知っているもの、$A\subseteq \mathbb{R}$とか$A\subseteq\mathbb{R}^n$とか
%\item $\emm{\theta \in \Theta}\colon$ パラメータ、母数、知りたいもの。
%$\Theta\subseteq\mathbb{R}$とか$\Theta\subseteq\mathbb{R}^n$とか。
%\item $s\in B$: パラメータ、推定したいもの(今日は離散の場合を考える。連続のときは$\theta$という記号を使うことが多い)
\item $\theta\in \Theta$: パラメータ、\emm{母数}、知りたいもの、$\Theta\subseteq\mathbb{R}$とか$\Theta\subseteq\mathbb{R}^n$とか
\item $p(\theta)$: 事前確率密度関数
\item $p(x \mid \theta)$: 尤度(条件付き確率密度関数)
\item $p(\theta \mid x)\coloneq p(x\mid \theta)p(\theta) / p(x)$: 事後確率密度関数
\end{itemize}
\end{frame}

\begin{frame}{誤差関数}
$\widehat{\theta}\colon A\to \Theta$
\begin{align*}
\expt{d(\widehat{\theta}(X), \theta)}
&=
\int_{A\times\Theta} d(\widehat{\theta}(x),\theta) p(x\mid\theta) p(\theta) \mathrm{d}x\mathrm{d}\theta
\end{align*}
\begin{align*}
\expt{(\widehat{\theta}(X)-\theta)^2}
&= \expt{\left(\widehat{\theta}(X)-\expt{\widehat{\theta}(X)}+\expt{\widehat{\theta}(X)}-\theta\right)^2}\\
&= \expt{\left(\widehat{\theta}(X)-\expt{\widehat{\theta}(X)}\right)^2}+\expt{\left(\expt{\widehat{\theta}(X)}-\theta\right)^2}\\
&\quad + 2\expt{\left(\widehat{\theta}(X)-\expt{\widehat{\theta}(X)}\right)\left(\expt{\widehat{\theta}(X)}-\theta\right)}\\
&= \var{\widehat{\theta}(X)}+\expt{\left(\expt{\widehat{\theta}(X)}-\theta\right)^2}
%\expt{\widehat{\theta}(X)^2 - 2\widehat{\theta}(X)\theta + \theta^2}
\end{align*}
\end{frame}

\begin{frame}{課題}
\small
$A=\{0,\,1\},\, B=\left\{p_0=\frac12,\,p_1=\frac13\right\}$とし、$k\in\{0,\,1\}$について
\begin{align*}
\Pr(X = 0 \mid S = p_k) &= 1-p_k&
\Pr(X = 1 \mid S = p_k) &= p_k
\end{align*}
とする。
このとき、 $\eta> 0,\,\kappa\in[0,1]$について、以下の尤度比検定関数$E(x)$で仮説検定することを考える。
\begin{align*}
E(x)&=
\begin{cases}
\vspace{.5em}
0&\text{if } \frac{\Pr(X = x\mid S = p_0)}{\Pr(X = x\mid S = p_1)} > \eta\\\vspace{.5em}
1&\text{if } \frac{\Pr(X = x\mid S = p_0)}{\Pr(X = x\mid S = p_1)} < \eta\\
\kappa&\text{otherwise.}
\end{cases}
\end{align*}
以下の問に答えよ。
\begin{enumerate}
\setlength{\itemsep}{1em}
\item $\eta=1$のとき$\alpha_E$, $\beta_E$をもとめよ。
\item 一般の$\eta>0$と$\kappa\in[0,1]$について、誤り確率$\alpha_E$, $\beta_E$をもとめよ。$\eta$の値で場合分けしてもとめること。
\item 任意の検定関数$E\colon A\to[0,1]$で実現可能な$(\alpha_E,\, \beta_E)$の範囲を図示せよ。
\end{enumerate}
\end{frame}


\end{document}
