\documentclass[lualatex,handout]{beamer}
\setbeamertemplate{footline}[frame number]
\usepackage{luatexja}
\usepackage{amsmath,amssymb}

\usepackage{thm-restate}

%\usetheme{Berlin}
\usecolortheme{rose}

\usepackage{tikz}
\usepackage{pgfplots}
\pgfplotsset{compat=1.18}

%\usepackage[haranoaji]{luatexja-preset}
\usepackage[deluxe,ipaex]{luatexja-preset}
\renewcommand{\kanjifamilydefault}{\gtdefault}
%\setmainjfont{HaranoAjiGothic-Regular}

\usepackage{unicode-math}
%\setmathfont{Fira Math}
\setmathfont{STIX Two Math}
\setmathfont{STIX Two Math}[range=bfup/{Latin,latin,num,Greek,greek}]
\setmathfont{STIX Two Math}[range=bfit/{Latin,latin}]
\setmathfont{STIX Two Math}[range={"0000-"FFFF}]
\setmathrm{STIX Two Math}[StylisticSet=8]

%\usefonttheme{professionalfonts}

\usepackage{luacolor}

\newcommand{\mycolor}[2]{%
  \begingroup
  \colorlet{currentcolor}{.}%
  \color{#1}#2%
  \color{currentcolor}%
  \endgroup
}
\newcommand{\emm}[1]{\mycolor{red}{#1}}
\newcommand{\expt}[1]{\mathbb{E}\left[#1\right]}
\newcommand{\var}[1]{\mathbb{V}\left[#1\right]}
\newcommand{\cov}[1]{\mathsf{Cov}\left[#1\right]}
\newcommand{\vc}[1]{\mathsf{Var}\left[#1\right]}

\DeclareMathOperator*{\argmax}{arg\,max}


\usepackage{xspace}
%\usepackage{bm}
%\newcommand\bm[1]{{\mathbf{#1}}}
\newcommand\defiff{\stackrel{\text{def}}{\iff}}
\newcommand\dx{{\,\mathrm{d}x}}
\newcommand\KL[2]{D\left(#1\,\|\,#2\right)}
\newcommand\dtv{d_{\mathrm{TV}}}

\theoremstyle{definition}

\title{確率・統計基礎: 点推定、十分統計量}
\author{森 立平}
\date{}



\begin{document}
\begin{frame}[plain]
\maketitle
\end{frame}



\begin{frame}{設定}
\begin{itemize}
\setlength{\itemsep}{2em}
\item $x\in \mathcal{X}$: データ、知っているもの。
%\item $\emm{\theta \in \Theta}\colon$ パラメータ、母数、知りたいもの。
%$\Theta\subseteq\mathbb{R}$とか$\Theta\subseteq\mathbb{R}^n$とか。
%\item $s\in B$: パラメータ、推定したいもの(今日は離散の場合を考える。連続のときは$\theta$という記号を使うことが多い)
\item $\theta\in \Theta$: \emm{パラメータ、母数}、知りたいもの、$\Theta\subseteq\mathbb{R}$とか$\Theta\subseteq\mathbb{R}^n$とか
\item $p(x \mid \theta)$ or $\Pr(X=x\mid\theta)$: 尤度(条件付き確率密度関数 or 条件付き確率質量関数)
\item $p(\theta)$: 事前確率密度関数、仮定する場合もある(ベイズ推定)が今回は仮定しない。
\end{itemize}
\end{frame}

%\begin{frame}{確率分布の族}
%a
%\end{frame}

\begin{frame}{推定したいものの例}
$g\colon \Theta\to\mathbb{R}$

\vspace{1em}
\begin{itemize}
\setlength{\itemsep}{2em}
\item $g(\theta) = \theta$.
\item $g(\theta) = \expt{X\mid\theta}$.
\item $g(\theta) = \var{X\mid\theta}$.
\item $g(\theta) = \expt{s(X)\mid\theta}$\qquad for some $s\colon\mathcal{X}\to\mathbb{R}$.
\end{itemize}
\end{frame}

\begin{frame}{損失関数とリスク関数}
\small

\begin{definition}[損失関数、リスク関数]
以下を満たす関数$L\colon \Theta\times\mathbb{R}\to \mathbb{R}$を$g(\theta)$の\emm{損失関数}という。
\begin{itemize}
\setlength{\itemsep}{2em}
\item 非負性
\begin{align*}
L(\theta,\, d) \ge 0\qquad \forall \theta\in\Theta,\,d\in\mathbb{R}.
\end{align*}
\item 正しいときはゼロ
\begin{align*}
L(\theta,\, g(\theta)) = 0\qquad \forall \theta\in\Theta.
\end{align*}
\end{itemize}
また、対応する\emm{リスク関数}$R\colon\Theta\times(\mathcal{X}\to\mathbb{R})\to\mathbb{R}$を以下で定義する。
\begin{align*}
R(\theta,\, \delta) \coloneq \expt{L(\theta,\, \delta(X))\mid \theta}.
\end{align*}
この期待値は$X$に関して取るものであるが、$\delta$がランダムである場合はそのランダムネスについても期待値を取る。
\end{definition}
\end{frame}

\begin{frame}{損失関数の例}
\small
\begin{itemize}
\setlength{\itemsep}{2em}
\item \emm{0-1損失関数} $L(\theta, d) = \mathbb{1}_{\{d \ne g(\theta)\}}$.
このとき、$R(\theta, \delta)=\Pr(\delta(X)\ne g(\theta)\mid\theta)$
\item \emm{二乗誤差損失} $L(\theta, d) = (g(\theta)- d)^2$.
このとき、$R(\theta, \delta)=\expt{(g(\theta) - \delta(X))^2\mid\theta}$.
この$R(\theta, \delta)$を\emm{平均二乗誤差}(MSE; meaned squared error)という。
\end{itemize}
\end{frame}

\begin{frame}{連続確率変数の条件付確率}
$Y$が離散確率変数の場合、
\begin{align*}
\Pr(X\in A\mid Y=y) &= \frac{\Pr(X\in A,\,Y=y)}{\Pr(Y=y)}
\end{align*}

\vspace{2em}
$Y$が連続確率変数の場合、
\begin{align*}
\Pr(X\in A\mid Y=y) &\coloneq \lim_{\epsilon\to0}\frac{\Pr(X\in A,\,Y\in[y,\,y+\epsilon])}{\Pr(Y\in[y,\,y+\epsilon])}
\end{align*}
\end{frame}

\begin{frame}{十分統計量}
\small
\begin{definition}[十分統計量]
$\mathcal{X}$上の確率分布族$p_\theta$について$T\colon \mathcal{X}\to \mathcal{Y}$が\emm{十分統計量}$\defiff$
任意の$A\subseteq \mathcal{X}$と$t\in\mathcal{Y}$について
\begin{align*}
%p(X\mid T) &\coloneq \frac{p()}{p(T)}
\Pr(X\in A\mid T(X)=t,\,\theta)
%&\qquad\text{$\mathcal{X}$と$\mathcal{Y}$が離散の場合}\\
%\Pr_\theta(X\in U\mid T(X)\in V) &= \frac{\int_{U\cap T^{-1}(V)} p_\theta(x)\mathrm{d}x}{\int_{T^{-1}(V)} p_\theta(x)\mathrm{d}x}
%p_\theta(x\mid t) &\coloneq \frac{p_\theta(x,t)}{{\int_{T^{-1}(\{t\})} p_\theta(x)\mathrm{d}x}}
%p_\theta(x\mid t) \coloneq 
%\begin{cases}
%\frac{p_\theta(x)}{q(t)}&\text{if } T(x) = t
%\end{cases}
%&\qquad\text{$\mathcal{X}$と$\mathcal{Y}$が連続の場合}
%\Pr(X\in A\mid T(x) = t) &=
\end{align*}
が$\theta$に依存しない。
\end{definition}
「\emm{$T(X)$は$X$に含まれている$\theta$の情報をすべて含む}」というのが直感的な意味。

\vspace{1em}
基本的には$X$の余計な部分を削ぎ落して$\theta$の推定に必要な部分だけを取り出したものを$T(X)$としたい。

\vspace{1em}
$T(x)=x$は自明な十分統計量であるが、余計な部分を削ぎ落していないので意味がない。
\end{frame}

\begin{frame}{十分統計量は推定において十分}
\footnotesize
\begin{theorem}
任意の推定関数$\delta\colon\mathcal{X}\to\mathbb{R}$と十分統計量$T\colon\mathcal{X}\to\mathcal{Y}$について、あるランダムな$\eta\colon\mathcal{Y}\to\mathbb{R}$が存在して、
推定関数$\eta\circ T$は同じリスク関数の値をとる。
\end{theorem}
\begin{proof}
\vspace{-1em}
\begin{align*}
R(\theta, \delta) &= \expt{L(\theta, \delta(X))\mid\theta}\\
 &= \sum_{x}\Pr(X=x\mid\theta) L(\theta, \delta(x))\\
 &= \sum_{x}\Pr(X=x,\,T(X)=T(x)\mid\theta) L(\theta, \delta(x))\\
 &= \sum_{x}\Pr(X=x\mid T(X)=T(x),\,\emm{\theta})\Pr(T(X)=T(x)\mid\theta) L(\theta, \delta(x))\\
 &= \sum_{x}\Pr(X=x\mid T(X)=T(x))\Pr(T(X)=T(x)\mid\theta) L(\theta, \delta(x))\\
 &= \sum_{t}\Pr(T(X)=t\mid\theta)\emm{\sum_{x\in T^{-1}(t)}\Pr(X=x\mid T(X)=t) L(\theta, \delta(x))}.
\end{align*}
\begin{align*}
\eta(t) &= \delta(x) \qquad\text{with probability $\Pr(X=x\mid T(X)=t)$}
\end{align*}
とおけば同じリスク関数の値になる。
\end{proof}
\end{frame}

\begin{frame}{十分統計量の例 1/3}
\small
 $X_1,\dotsc, X_N$が独立同分布なベルヌーイ分布$\mathrm{Ber}(\emm{p})$
\begin{align*}
\Pr(X_1=x_1,\dotsc, X_N=x_N\mid \emm{p}) &=\emm{p}^{x_1+\dotsc+x_n}(1-\emm{p})^{N-(x_1+\dotsb+x_N)}
\end{align*}
に従っているとき、$T(x_1,\dotsc,x_n)=\emm{\sum_{k=1}^Nx_k}$は\emm{$\emm{p}$の十分統計量}である。
実際、
\begin{align*}
\Pr\left(T(X_1,\dotsc,X_N)=t\mid \emm{p}\right)&=\Pr\left(\sum_k T_k=t\mid \emm{p}\right)\\
&= \binom{N}{t} \emm{p}^t (1-\emm{p})^{N-t}.
\end{align*}
よって$\sum_k x_k=t$のとき、
\begin{align*}
\Pr\left(X_1=x_1,\dotsc,X_N=x_N\mid T(X_1,\dotsc,X_N)=t,\,\emm{p}\right)&=\frac{\emm{p}^t(1-\emm{p})^{N-t}}{\binom{N}{t} \emm{p}^t (1-\emm{p})^{N-t}}\\
&=\frac1{\binom{N}{t}}.
\end{align*}
これは$\emm{p}$に依存していない。
\end{frame}

\begin{frame}{十分統計量の例 2/3}
\small
%\begin{itemize}
%\setlength{\itemsep}{2em}
%\item
 $X_1,\,X_2$が独立同分布なポアソン分布
\begin{align*}
\Pr(X_1=x_1,\,X_2=x_2\mid\emm{\lambda}) &=\frac{\emm{\lambda}^{x_1+x_2}}{x_1!x_2!}\mathrm{e}^{-2\emm{\lambda}}
\end{align*}
に従っているとき、$T(x_1,x_2)=\emm{x_1+x_2}$は\emm{$\lambda$の十分統計量}である。
実際、
%
\begin{align*}
\Pr(T(X_1,\,X_2)=t\mid\emm{\lambda}) &= \sum_{x=0}^t\Pr(X_1=x,\, X_2=t-x\mid\emm{\lambda})\\
&=\sum_{x=0}^t \frac{\emm{\lambda}^{t}}{x!(t-x)!}\mathrm{e}^{-2\emm{\lambda}}\\
&=\frac1{t!}(2\emm{\lambda})^t\mathrm{e}^{-2\emm{\lambda}}.
\end{align*}
よって$x_1+x_2=t$のとき、
\begin{align*}
\Pr(X_1=x_1,\,X_2=x_2\mid T(X_1,\,X_2) = t,\,\emm{\lambda}) &=\frac{\emm{\lambda}^{x_1+x_2}}{x_1!x_2!}\mathrm{e}^{-2\emm{\lambda}}
t!(2\emm{\lambda})^{-t}\mathrm{e}^{2\emm{\lambda}}\\
&=\binom{t}{x_1}2^{-t}.
\end{align*}
これは$\emm{\lambda}$に依存してない($X_1\sim\mathrm{Binom}(t, \frac12)$)。
%\item $p_\theta(x) = \frac1{\sqrt{2\pi \theta^2}} \mathrm{e}^{-\frac{x^2}{2\theta}}$のとき、$T(x)=x^2$は十分統計量。
%\begin{align*}
%\Pr_\theta(X\in A\mid T(X) = t) &= \lim_{\epsilon\to 0} \frac{\Pr(X\in A,\, X^2\in[t,\,t+\epsilon])}{\Pr(X^2\in[t,\,t+\epsilon])}
%\end{align*}
%\end{itemize}
\end{frame}

\begin{frame}{十分統計量の例 3/3}
\small
$X$が従っている分布が対称であるとする。
つまり
\begin{align*}
\Pr(X\in A\mid\theta) &= \Pr(-X\in A\mid\theta)
\end{align*}
このとき、$T(x)=|x|$は$\theta$の十分統計量である。

任意の$\delta >0$について
\begin{align*}
\Pr(X\in [t,\,t+\delta]\mid T(X) = t,\,\theta)
 &= \lim_{\epsilon\to0} \frac{\Pr(X\in [t,\,t+\delta],\, |X|\in[t,\,t+\epsilon]\mid\theta)}{\Pr(|X|\in[t,\,t+\epsilon]\mid\theta)}\\
 &= \lim_{\epsilon\to0} \frac{\Pr(-X\in [t,\,t+\delta],\, |X|\in[t,\,t+\epsilon]\mid\theta)}{\Pr(|X|\in[t,\,t+\epsilon]\mid\theta)}\\
 &= \lim_{\epsilon\to0} \frac{\Pr(X\in [-t-\delta,\,-t],\, |X|\in[t,\,t+\epsilon]\mid\theta)}{\Pr(|X|\in[t,\,t+\epsilon]\mid\theta)}\\
 &= \Pr(X\in [-t-\delta,\,-t]\mid T(X) = t,\,\theta)
%&=\frac12
\end{align*}
一方で$\Pr(X\notin ([t,\,t+\delta] \cup [-t-\delta,\,-t])\mid T(X)=t,\,\theta)=0$なので、
よって、$\Pr(X=t\mid T(X)=t,\,\theta)=\Pr(X=-t\mid T(X)=t,\,\theta)=\frac12$である。
\end{frame}

\if0
\begin{frame}{十分統計量の例 2/3}
$X_1,\dotsc,X_N$が独立同分布な一様分布$U(0,\theta)$に従っているとする。
つまり$a_1,\dotsc,a_N\in[0,\,\theta]$について、
\begin{align*}
\Pr(X_1\le a_1,\dotsc, X_N\le a_N)&= \prod_{k=1}^N\frac{a_k}{\theta}
\end{align*}
このとき、$T(x_1,\dotsc,x_n) = \max_k x_k$は$\theta$の十分統計量である。

\begin{align*}
\Pr(T(X_1,\dotsc,X_N)\le a)&= \Pr(\max_k X_k\le a)\\
&= \Pr(X_1\le a,\dotsc, X_N\le a)\\
&= \left(\frac{a}{\theta}\right)^N
\end{align*}

\begin{align*}
\Pr(X_1\le a_1,\dotsc, X_N\le a_N\mid T(X_1,\dotsc,X_N)=a)&= \lim_{\epsilon\to0}\frac{\prod_{k=1}^N\frac{a_k}{\theta}}
{\left(\frac{a+\epsilon}{\theta}\right)^N-\left(\frac{a}{\theta}\right)^N}
\end{align*}
\end{frame}
\fi


\begin{frame}{分解定理 1/2}
\footnotesize
\begin{theorem}[\small フィッシャー・ネイマンの分解定理]
確率分布族$(p_\theta)_\theta$について以下は同値。
\begin{enumerate}
\item $T\colon\mathcal{X}\to\mathcal{Y}$が$(p_\theta)_\theta$の十分統計量である。
\item ある$u\colon\mathcal{X}\to\mathbb{R}$, $(v_\theta\colon \mathcal{Y}\to\mathbb{R})_\theta$が存在して
\begin{align*}
p_\theta(x) &= u(x) v_\theta(T(x)).
\end{align*}
\end{enumerate}
\end{theorem}
\begin{proof}[\small 離散の場合($1\to 2$)]
\begin{align*}
\Pr(X=x\mid \theta) &= \Pr(X=x, T(X) = T(x)\mid \theta)\\
&= \emm{\Pr(X=x\mid T(X) = T(x),\,\theta)}\Pr(T(X)=T(x)\mid\theta)\\
&= \emm{u(x)}v_\theta(T(x)).
\end{align*}
%$2\to1\colon$
%\begin{align*}
%\Pr(T(x)=t\mid\theta)&=\sum_{x\in T^{-1}(t)} \Pr(X=x\mid \theta)\\
%&=\sum_{x\in T^{-1}(t)} u(x)v_\theta(T(x))\\
%&=\left(\sum_{x\in T^{-1}(t)}\right) u(x)v_\theta(t)
%\end{align*}
\end{proof}
\end{frame}

\begin{frame}{分解定理 2/2}
\footnotesize
\begin{theorem}[\small フィッシャー・ネイマンの分解定理]
確率分布族$(p_\theta)_\theta$について以下は同値。
\begin{enumerate}
\item $T\colon\mathcal{X}\to\mathcal{Y}$が$(p_\theta)_\theta$の十分統計量である。
\item ある$u\colon\mathcal{X}\to\mathbb{R}$, $(v_\theta\colon \mathcal{Y}\to\mathbb{R})_\theta$が存在して
\begin{align*}
p_\theta(x) &= u(x) v_\theta(T(x)).
\end{align*}
\end{enumerate}
\end{theorem}
\begin{proof}[\small 離散の場合($2\to 1$)]

\vspace{-1em}
\begin{align*}
\Pr(T(x)=t\mid\theta)&=\sum_{x\in T^{-1}(t)} \Pr(X=x\mid \theta)
=\sum_{x\in T^{-1}(t)} u(x)v_\theta(T(x))\\
&=\left(\sum_{x\in T^{-1}(t)} u(x)\right)v_\theta(t).\\
\Pr(X=x\mid T(X)=T(x),\, \theta)&=\frac{u(x)v_\theta(T(x))}{\left(\sum_{x'\in T^{-1}(T(x))} u(x')\right)v_\theta(T(x))}\\
&=\frac{u(x)}{\sum_{x'\in T^{-1}(T(x))} u(x')}. \qquad\text{($\theta$に依存してない)}
\end{align*}
\end{proof}
\end{frame}

\begin{frame}{尤度比は十分統計量}
\small
\begin{lemma}
パラメータ空間$\Theta=\{\theta_0,\,\theta_1,\dotsc,\theta_k\}$が有限集合とする。
\begin{align*}
T(x) &= \left(\frac{\Pr(X=x\mid\theta_1)}{\Pr(X=x\mid\theta_0)},\, \frac{\Pr(X=x\mid\theta_2)}{\Pr(X=x\mid\theta_0)},\dotsc,\,\frac{\Pr(X=x\mid\theta_k)}{\Pr(X=x\mid\theta_0)}\right)
\end{align*}
は十分統計量である。
\end{lemma}
\begin{proof}
\vspace{-1em}
\begin{align*}
\Pr(X=x\mid\theta_\ell)&=
\begin{cases}
\Pr(X=x\mid\theta_0)&\text{if } \ell = 0\\
\Pr(X=x\mid\theta_0) T(x)_\ell&\text{otherwise.}
\end{cases}
\end{align*}
よって
\begin{align*}
u(x) &= \Pr(X=x\mid\theta_0),&
v_{\theta_\ell}(t) &=
\begin{cases}
1&\text{if } \ell = 0\\
t_\ell&\text{otherwise.}
\end{cases}
\end{align*}
とおくと$\Pr(X=x\mid\theta_\ell)=u(x)v_{\theta_\ell}(T(x))$.
\end{proof}
\end{frame}

\begin{frame}{最小十分統計量}
\footnotesize
\begin{definition}[\emm{最小}十分統計量]
十分統計量$T\colon\mathcal{X}\to\mathcal{Y}$が\emm{最小}十分統計量
$\defiff$
任意の十分統計量$U\colon\mathcal{X}\to\mathcal{Z}$についてある関数$V\colon\mathcal{Z}\to\mathcal{Y}$が存在して$T(x) = V(U(x))$.
\end{definition}
\begin{theorem}
パラメータ空間$\Theta=\{\theta_0,\,\theta_1,\dotsc,\theta_k\}$が有限集合とする。
\begin{align*}
T(x) &= \left(\frac{\Pr(X=x\mid\theta_1)}{\Pr(X=x\mid\theta_0)},\, \frac{\Pr(X=x\mid\theta_2)}{\Pr(X=x\mid\theta_0)},\dotsc,\,\frac{\Pr(X=x\mid\theta_k)}{\Pr(X=x\mid\theta_0)}\right)
\end{align*}
は\emm{最小}十分統計量である。
\end{theorem}
\vspace{-.7em}
\begin{proof}
$U\colon\mathcal{X}\to\mathcal{Z}$を十分統計量とする。分解定理より、ある$u$と$v_\theta$が存在して
\begin{align*}
\Pr(X=x\mid\theta_t)&= u(x)v_{\theta_t}(U(x)).
\end{align*}
\vspace{-.5em}
このとき、
\vspace{-.5em}
\begin{align*}
\frac{\Pr(X=x\mid\theta_t)}{\Pr(X=x\mid\theta_0)}&= \frac{v_{\theta_t}(U(x))}{v_{\theta_0}(U(x))}
\end{align*}
よって$T(x)=V(U(x))$と表すことができる。
\end{proof}
\end{frame}

\begin{frame}{ラオ--ブラックウェルの定理}
\small
\begin{theorem}[ラオ--ブラックウェルの定理]
損失関数$L(\theta, d)$が$d$について\emm{凸関数}と仮定する。
任意の推定関数$\delta\colon\mathcal{X}\to\mathbb{R}$と十分統計量$T\colon\mathcal{X}\to\mathcal{Y}$についてある(ランデムでない)推定関数$\eta\colon\mathcal{Y}\to\mathbb{R}$が存在して、$R(\theta,\, \eta\circ T)\le R(\theta,\,\delta)$である。
\end{theorem}
\begin{proof}
\begin{align*}
\eta(t) &=\expt{\delta(X)\mid T(X)=t}
\end{align*}
とおく。
このとき、
\begin{align*}
R(\theta,\,\eta\circ T)
% &= \sum_{x} \Pr(X=x\mid\theta) L(\theta, \delta'(T(x)))
&= \sum_{t} \Pr(T(X)=t\mid\theta) L(\theta, \eta(t))\\
&= \sum_{t} \Pr(T(X)=t\mid\theta) L(\theta, \expt{\delta(X)\mid T(X)=t})\\
&\le \sum_{t} \Pr(T(X)=t\mid\theta) \expt{L(\theta, \delta(X))\mid T(X)=t}\qquad\text{(イェンセン)}\\
&= \expt{L(\theta, \delta(X))\mid \theta} = R(\theta,\,\delta).\qedhere
\end{align*}
\end{proof}
\end{frame}

\begin{frame}{決定的な推定}
\begin{lemma}[脱ランダム化]
損失関数$L(\theta, d)$が$d$について\emm{凸関数}と仮定する。
任意の(ランダムでもよい)推定関数$\delta\colon\mathcal{X}\to\mathbb{R}$ついてあるランデムでない推定関数$\delta'\colon\mathcal{X}\to\mathbb{R}$が存在して、$R(\theta,\, \delta')\le R(\theta,\,\delta)$である。
\end{lemma}
\begin{proof}
$\delta'$を以下で定義する。
\begin{align*}
\delta'(x) &= \expt{\delta(x)}
\end{align*}
ここで$\expt{\cdot}$は推定関数の持つランダムネスについての期待値である。
目的の不等式はイェンセンの不等式により得られる。
\end{proof}
\end{frame}

\if0
\begin{frame}{二乗誤差}
\small
\begin{align*}
\mathrm{MSE}(\theta,\,\delta)&= \expt{(\delta(X)-g(\theta))^2\mid\theta}\\
&= \expt{\left(\delta(X)-\expt{\delta(X)\mid\theta}+\expt{\delta(X)\mid\theta}-g(\theta)\right)^2\mid\theta}\\
&= \expt{\left(\delta(X)-\expt{\delta(X)\mid\theta}\right)^2\mid\theta}+\expt{\left(\expt{\delta(X)\mid\theta}-g(\theta)\right)^2\mid\theta}\\
&\quad + 2\expt{\left(\delta(X)-\expt{\delta(X)\mid\theta}\right)\left(\expt{\delta(X)\mid\theta}-g(\theta)\right)\mid\theta}\\
&= \var{\delta(X)\mid\theta}+\left(\expt{\delta(X)\mid\theta}-g(\theta)\right)^2\\
&= \var{\delta(X)\mid\theta}+\mathrm{Bias}_g(\delta\mid\theta)^2
%\expt{\widehat{\theta}(X)^2 - 2\widehat{\theta}(X)\theta + \theta^2}
\end{align*}
\begin{align*}
\mathrm{Bias}_g(\delta\mid\theta)&\coloneq\expt{\delta(X)\mid\theta}-g(\theta)
\end{align*}
\begin{definition}[不偏推定量]
推定量$\delta\colon \mathcal{X}\to\mathbb{R}$が以下を満たすとき、$g(\theta)$の\emm{不偏推定量}であるという。
\begin{align*}
\mathrm{Bias}_g(\delta\mid\theta) &= 0 \qquad\forall \theta\in\Theta.
\end{align*}
\end{definition}
\end{frame}

\begin{frame}{推定量の平均}
\footnotesize
\begin{lemma}
$\widehat{\theta}_1,\,\widehat{\theta}_2,\dotsc,\widehat{\theta}_n\colon A\to\Theta$が不偏推定量のとき、
\begin{align*}
\widehat{\theta}^* &\coloneq \frac1n\sum_{k=1}^n\widehat{\theta}_k
\end{align*}
は不偏推定量で
\begin{align*}
\var{\widehat{\theta}^*(X)\mid\theta} &\le\frac1n\sum_{k=1}^n\var{\widehat{\theta}_k(X)\mid\theta}
\qquad\forall\theta\in\Theta.
\end{align*}
\end{lemma}
\begin{proof}
$\widehat{\theta}^*$が不偏推定量であることは期待値の線形性から明らか。
\begin{align*}
\var{\widehat{\theta}^*(X)\mid\theta} &=
\expt{\left(\frac1n\sum_{k=1}^n\widehat{\theta}_k(X) - \theta\right)^2\mid\theta}\\
&\le
\frac1n\sum_{k=1}^n\expt{\left(\widehat{\theta}_k(X) - \theta\right)^2\mid\theta}\quad\text{(イェンセンの不等式)}\qedhere
%\frac1{n^2} \left(\sum_{k=1}^n \var{\widehat{\theta}_k(X)\mid\theta} + 2\sum_{k<\ell} \cov{\widehat{\theta}_k,\,\widehat{\theta}_\ell}\right)
\end{align*}
\end{proof}
\end{frame}

\begin{frame}{対称な不偏推定量}
\small
\begin{align*}
p(\symbf{x}\mid \theta) &= \prod_{k=1}^N p(x_k\mid \theta).
\end{align*}
\begin{lemma}[対称な不偏推定量]
任意の不偏推定量$\widehat{\theta}\colon A^N\to\Theta$について、ある対称な不偏推定量$\widehat{\theta}^*$が存在して
\begin{align*}
\var{\widehat{\theta}^*(\symbf{X})\mid\theta} &\le \var{\widehat{\theta}(\symbf{X})\mid\theta}
\qquad\forall\theta\in\Theta.
\end{align*}
\end{lemma}
\begin{proof}
\begin{align*}
\widehat{\theta}^*(x_1,\dotsc,x_N) &\coloneq \frac1{N!} \sum_{\sigma\in S_N} \widehat{\theta}\left(x_{\sigma(1)},\dotsc,x_{\sigma(N)}\right)
\end{align*}
と定義すればよい。
\end{proof}
\end{frame}
\fi

\if0
\begin{frame}{加法的ノイズのモデル}
\footnotesize
\begin{align*}
X &= \theta + Z
\end{align*}
\begin{itemize}
\item $\expt{Z} = 0$.
\item $\var{Z} < \infty$.
\end{itemize}
\begin{theorem}
$\widehat{\theta}^*$が$Z$の分布によらず対称な不偏推定量となるとき、$\widehat{\theta}^*$は\emm{標本平均}
\begin{align*}
\widehat{\theta}^*(x_1,\dotsc,x_n)&\coloneq \frac1n \sum_{k=1}^n x_k
\end{align*}
である。
\end{theorem}
\begin{proof}
$n=1$のとき、
$\Pr(Z=0)=1$と仮定すると、
\begin{align*}
\theta&= \expt{\widehat{\theta}^*(X)\mid\theta}
= \widehat{\theta}^*(\theta)
\end{align*}
が任意の$\theta\in\Theta$について成り立つ。よって$\widehat{\theta}^*(x)=x$である。
\end{proof}
\end{frame}

\begin{frame}{加法的ノイズのモデル}
\small
ある$a\ne 0$と$p\in[0,1]$について、ノイズの分布を
\begin{align*}
\Pr(Z = a(p-1)) &= p&
\Pr(Z = ap) &= 1-p
\end{align*}
とすると$\expt{Z}=0$, $\var{Z}<\infty$である。

$n=2$のとき、
任意の$a\ne 0$と$p\in[0,1]$について
\begin{align*}
%\expt{\widehat{\theta}^*(\symbf{X})\mid\theta} &=
\theta&= \expt{\widehat{\theta}^*(X_1,X_2)\mid\theta}\\
 &= p^2 \widehat{\theta}^*(a(p-1),a(p-1)) + (1-p)^2 \widehat{\theta}^*(ap, ap)\\
&\quad + p(1-p) \widehat{\theta}^*(a(p-1), ap) + (1-p)p\widehat{\theta}^*(ap, a(p-1))\\
 &= p^2 a(p-1) + (1-p)^2 ap + 2p(1-p) \widehat{\theta}^*(ap, a(p-1))
\end{align*}
が成り立つ。
\end{frame}
\fi

\begin{frame}{課題}
\small
以下の問に答えよ。
\begin{enumerate}
\setlength{\itemsep}{1em}
\item 
\begin{align*}
p(\symbf{x}\mid\theta)&= \frac1{\sqrt{(2\pi)^N}}\mathrm{e}^{-\sum_{k=1}^N\frac{(x_k-\theta)^2}2}
\end{align*}
のとき、以下の$T\colon\mathbb{R}^N\to\mathbb{R}$が十分統計量かどうか示せ。結論だけでなく理由も示すこと。
\begin{align*}
T(\symbf{x})&= \sum_{k=1}^N x_k.
\end{align*}
\item
\begin{align*}
\Pr(X_1=x_1,\dotsc,X_N=x_N\mid\lambda)&= \frac1{\prod_{k=1}^N x_k!}\lambda^{\sum_{k=1}^N x_k} \mathrm{e}^{-N\lambda}
\end{align*}
のとき、推定関数$\delta(\symbf{x})=x_1$と十分統計量$T(\symbf{x})=\sum_{k=1}^Nx_k$を用いてラオ--ブラックウェルの定理に基づいて、よりよい推定関数$\eta\circ T$をもとめよ。
\end{enumerate}
\end{frame}


\end{document}
