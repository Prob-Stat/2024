\documentclass[lualatex,handout]{beamer}
\setbeamertemplate{footline}[frame number]
\usepackage{luatexja}
\usepackage{amsmath,amssymb}

\usepackage{thm-restate}

%\usetheme{Berlin}
\usecolortheme{rose}

\usepackage{tikz}
\usepackage{pgfplots}
\pgfplotsset{compat=1.18}

%\usepackage[haranoaji]{luatexja-preset}
\usepackage[deluxe,ipaex]{luatexja-preset}
\renewcommand{\kanjifamilydefault}{\gtdefault}
%\setmainjfont{HaranoAjiGothic-Regular}

\usepackage{unicode-math}
%\setmathfont{Fira Math}
\setmathfont{STIX Two Math}
\setmathfont{STIX Two Math}[range=bfup/{Latin,latin,num,Greek,greek}]
\setmathfont{STIX Two Math}[range=bfit/{Latin,latin}]
\setmathfont{STIX Two Math}[range={"0000-"FFFF}]
\setmathrm{STIX Two Math}[StylisticSet=8]

%\usefonttheme{professionalfonts}

\usepackage{luacolor}

\newcommand{\mycolor}[2]{%
  \begingroup
  \colorlet{currentcolor}{.}%
  \color{#1}#2%
  \color{currentcolor}%
  \endgroup
}
\newcommand{\emm}[1]{\mycolor{red}{#1}}
\newcommand{\expt}[1]{\mathbb{E}\left[#1\right]}
\newcommand{\var}[1]{\mathbb{V}\left[#1\right]}
\newcommand{\cov}[1]{\mathsf{Cov}\left[#1\right]}
\newcommand{\vc}[1]{\mathsf{Var}\left[#1\right]}

\DeclareMathOperator*{\argmax}{arg\,max}


\usepackage{xspace}
%\usepackage{bm}
%\newcommand\bm[1]{{\mathbf{#1}}}
\newcommand\defiff{\stackrel{\text{def}}{\iff}}
\newcommand\dx{{\,\mathrm{d}x}}
\newcommand\KL[2]{D\left(#1\,\|\,#2\right)}
\newcommand\dtv{d_{\mathrm{TV}}}

\theoremstyle{definition}

\title{確率・統計基礎: 二値の推定}
\author{森 立平}
\date{}



\begin{document}
\begin{frame}[plain]
\maketitle
\end{frame}


\begin{frame}{統計の問題}
現実の問題では確率分布が未知であることがほとんど

\vspace{2em}
確率分布や統計量を推定するのが統計の主な問題

\vspace{2em}
一番単純な設定では確率分布は$p$か$q$のどちらかを当てる設定。
\end{frame}

\begin{frame}{ベイズ}
\begin{block}{問題}
確率$\lambda\in[0,1]$で確率分布$p_0\in\mathcal{P}$、確率$1-\lambda$で確率分布$p_1\in\mathcal{P}$
が選ばれ、その分布に従って$x\in A$が定まる。

\vspace{1em}
$x$を見て選ばれた分布が$p_0$か$p_1$か決定せよ。
\end{block}

\begin{example}
\begin{itemize}
\item $A=\{0,1\}^{100}$
\item $p_0$が成功確率$1/2$、試行回数$100$の二項分布
\item $p_1$が成功確率$1/3$、試行回数$100$の二項分布
\end{itemize}
\end{example}

\vspace{1em}
$p_0$と$p_1$を識別する関数$E\colon A\to\{0,1\}$とすると、その誤り確率は

\begin{align*}
%\min_{E\colon A\to \{0,1\}}
% \left(
\lambda \sum_{x\in E^{-1}(1)} p_0(x) + (1-\lambda) \sum_{x\in E^{-1}(0)}p_1(x)
% \right)
\end{align*}
\end{frame}

\begin{frame}{最小誤り確率}
\small
\begin{align*}
&\min_{E\colon A\to \{0,1\}}
 \left( \lambda \sum_{x\in E^{-1}(1)} p_0(x) + (1-\lambda) \sum_{x\in E^{-1}(0)}p_1(x) \right)\\
&=\min_{E\colon A\to \{0,1\}}
 \left( \lambda \sum_{x\in E^{-1}(1)} p_0(x) + (1-\lambda) \left(1-\sum_{x\in E^{-1}(1)}p_1(x) \right)\right)\\
&=\min_{E\colon A\to \{0,1\}}
 \left( (1-\lambda) + \sum_{x\in E^{-1}(1)} \left(\lambda p_0(x) - (1-\lambda) p_1(x) \right)\right)\\
&= (1-\lambda) + \min_{B\subseteq A} \left(\sum_{x\in B} \left(\lambda p_0(x) - (1-\lambda) p_1(x) \right)\right)\\
\end{align*}
ここで$\sum_{x\in B}$の部分は\emm{項が負のところだけ足す}ときに最小化されるので
\begin{align*}
&  (1-\lambda) + \frac{\left(\sum_{x\in A}\lambda p_0(x) - (1-\lambda) p_1(x) \right) - \sum_{x\in A}\left|\lambda p_0(x) - (1-\lambda) p_1(x) \right|}2\\
&=\frac{1-\emm{\sum_{x\in A}\left|\lambda p_0(x) - (1-\lambda) p_1(x) \right|}}2.
\end{align*}
\end{frame}

\begin{frame}{全変動距離}
最小誤り確率は
\begin{align*}
\frac{1-\emm{\sum_{x\in A}\left|\lambda p_0(x) - (1-\lambda) p_1(x) \right|}}2.
\end{align*}
特に$\lambda=1/2$のとき、
\begin{align*}
\frac{1-\emm{\frac12\sum_{x\in A}\left|p_0(x) - p_1(x) \right|}}2.
\end{align*}
\begin{definition}[全変動距離(Total variation distance)]
$p,\, q\in\mathcal{P}$についてその全変動距離$\dtv(p,\,q)$を以下で定義する。
\begin{align*}
\dtv(p,\, q) &\coloneq
\frac12\sum_{x\in A}\left|p(x) - q(x) \right|.
\end{align*}
\end{definition}
\end{frame}

\begin{frame}{最大事後確率}
\small
識別したい分布の数が二つではなく、それ以上(有限個)の場合を考える。

\begin{block}{問題}
確率$\lambda_s$で確率分布$p_s\in\mathcal{P}$が選ばれ、その分布に従って$x\in A$が定まる。

\vspace{1em}
$x$を見て選ばれた分布$p_s$の添字$s$を決定せよ。
\end{block}

$s$に対応する確率変数を$S$、$x$に対応する確率変数を$X$とおく。つまり
\begin{align*}
\Pr(S=s,\, X=x) &= \Pr(S = s) \Pr(X = x\mid S = s) = \lambda_s p_s(x).
\end{align*}
このとき、識別器$E\colon A\to \{1,2,\dotsc,s\}$の誤り確率は
\begin{align*}
\Pr(E(X) \ne S) &= 
%\sum_{s \in } \Pr(S = s)\Pr(E(X)\ne S\mid S = s)\\
\sum_{x \in A} \Pr(X = x)\Pr(E(X)\ne S\mid X = x)\\
&=\sum_{x \in A} \Pr(X = x)\Pr(E(x)\ne S\mid X = x)\\
&=\sum_{x \in A} \Pr(X = x) \sum_{s\ne E(x)} \Pr(S=s\mid X = x)
%\sum_{s \in S} \lambda_s \sum_{x \notin E^{-1}(s)} p_s(x)
\end{align*}
これを最小化したいので
$E(x)=\emm{\argmax_{s} \Pr(S=s\mid X=x)}$とするのが最適である。
\end{frame}

\begin{frame}{用語}
\begin{itemize}
\setlength{\itemsep}{2em}
\item $s$
\item $x$
\item $\Pr(S = s)$: 事前確率
\item $\Pr(X = x \mid S = s)$: 尤度
\item $\Pr(S = s \mid X = x)$: 事後確率
\end{itemize}
\end{frame}

\begin{frame}{漸近}
\begin{align*}
\Pr\left(\right)
\end{align*}
\end{frame}

\begin{frame}{課題}
\begin{itemize}
\item 二元エントロピー関数$h(p)\coloneq -p\log_2 p - (1-p)\log_2(1-p)\quad\text{for } p\in[0,1]$のグラフを描け。凸性に気をつけること。
%\item 確率変数$X\sim\mathrm{Multinom}(p)$について
%\begin{align*}
%\lim_{N\to\infty} \frac1N\log\Pr\left(\KL{P_{\symbf{X}}}{p}\ge a\right)
%\end{align*}
%をもとめよ。
%\item $\Gamma_a\coloneq\{q\in\mathcal{P}\mid \sum_k q_k k \ge a\}$
\end{itemize}
\end{frame}

\end{document}
