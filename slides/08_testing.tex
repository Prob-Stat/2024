\documentclass[lualatex,handout]{beamer}
\setbeamertemplate{footline}[frame number]
\usepackage{luatexja}
\usepackage{amsmath,amssymb}

\usepackage{thm-restate}

%\usetheme{Berlin}
\usecolortheme{rose}

\usepackage{tikz}
\usepackage{pgfplots}
\pgfplotsset{compat=1.18}

%\usepackage[haranoaji]{luatexja-preset}
\usepackage[deluxe,ipaex]{luatexja-preset}
\renewcommand{\kanjifamilydefault}{\gtdefault}
%\setmainjfont{HaranoAjiGothic-Regular}

\usepackage{unicode-math}
%\setmathfont{Fira Math}
\setmathfont{STIX Two Math}
\setmathfont{STIX Two Math}[range=bfup/{Latin,latin,num,Greek,greek}]
\setmathfont{STIX Two Math}[range=bfit/{Latin,latin}]
\setmathfont{STIX Two Math}[range={"0000-"FFFF}]
\setmathrm{STIX Two Math}[StylisticSet=8]

%\usefonttheme{professionalfonts}

\usepackage{luacolor}

\newcommand{\mycolor}[2]{%
  \begingroup
  \colorlet{currentcolor}{.}%
  \color{#1}#2%
  \color{currentcolor}%
  \endgroup
}
\newcommand{\emm}[1]{\mycolor{red}{#1}}
\newcommand{\expt}[1]{\mathbb{E}\left[#1\right]}
\newcommand{\var}[1]{\mathbb{V}\left[#1\right]}
\newcommand{\cov}[1]{\mathsf{Cov}\left[#1\right]}
\newcommand{\vc}[1]{\mathsf{Var}\left[#1\right]}


\usepackage{xspace}
%\usepackage{bm}
%\newcommand\bm[1]{{\mathbf{#1}}}
\newcommand\defiff{\stackrel{\text{def}}{\iff}}
\newcommand\dx{{\,\mathrm{d}x}}
\newcommand\KL[2]{D\left(#1\,\|\,#2\right)}

\theoremstyle{definition}

\title{確率・統計基礎: 二値の推定}
\author{森 立平}
\date{}



\begin{document}
\begin{frame}[plain]
\maketitle
\end{frame}


\begin{frame}{統計の問題}
現実の問題では確率分布が未知であることがほとんど

\vspace{2em}
確率分布や統計量を推定するのが統計の主な問題

\vspace{2em}
一番単純な設定では確率分布は$p$か$q$のどちらかを当てる設定。
\end{frame}

\begin{frame}{ベイズ}
確率$\lambda\in[0,1]$で確率分布$p\in\mathcal{P}$、確率$1-\lambda$で確率分布$q\in\mathcal{P}$
\end{frame}

\begin{frame}{課題}
\begin{itemize}
\item 二元エントロピー関数$h(p)\coloneq -p\log_2 p - (1-p)\log_2(1-p)\quad\text{for } p\in[0,1]$のグラフを描け。凸性に気をつけること。
%\item 確率変数$X\sim\mathrm{Multinom}(p)$について
%\begin{align*}
%\lim_{N\to\infty} \frac1N\log\Pr\left(\KL{P_{\symbf{X}}}{p}\ge a\right)
%\end{align*}
%をもとめよ。
%\item $\Gamma_a\coloneq\{q\in\mathcal{P}\mid \sum_k q_k k \ge a\}$
\end{itemize}
\end{frame}

\end{document}
