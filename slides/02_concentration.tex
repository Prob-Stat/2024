\documentclass[lualatex,handout]{beamer}
\usepackage{luatexja}
\usepackage{amsmath,amssymb}

\usepackage{tikz}

%\usepackage[haranoaji]{luatexja-preset}
\usepackage[deluxe,ipaex]{luatexja-preset}
\renewcommand{\kanjifamilydefault}{\gtdefault}
%\setmainjfont{HaranoAjiGothic-Regular}

\usepackage{unicode-math}
%\setmathfont{Fira Math}
%\setmathfont{STIX Two Math}
%\setmathfont{STIX Two Math}[range={up,it,bb,bbit,scr,bfscr},StylisticSet=8]
\setmathfont{XITS Math}
\setmathrm{XITS Math}[StylisticSet=8]
%\setmathfont{XITS Math}[StylisticSet=8]

%\usefonttheme{professionalfonts}

\usepackage{luacolor}

\newcommand{\emm}[1]{\textcolor{red}{#1}}
\newcommand{\expt}[1]{\mathbb{E}\left[#1\right]}
\newcommand{\var}[1]{\mathbb{V}\left[#1\right]}
\newcommand{\cov}[1]{\mathrm{Cov}\left[#1\right]}


\usepackage{xspace}
\newcommand\bm[1]{{\mathbf{#1}}}

\theoremstyle{definition}

\title{確率・統計基礎: 集中不等式}
\author{森 立平}
\date{}



\begin{document}
\begin{frame}[plain]
\maketitle
\end{frame}

\begin{frame}{集中不等式}
$N$個の独立な確率変数の期待値が平均周辺に集中するという

独立同分布(independently and identically distributed (i.i.d.))
\end{frame}


\begin{frame}{集中不等式}
\begin{theorem}[大数の弱法則]
確率変数$X$が分散を持つとする。$(X_t)_{t=1,\dotsc,}$が$X$と独立同分布とする。
このとき、任意の実数$\epsilon>0$について
\begin{align*}
\lim_{N\to\infty}\Pr\left(\left|\frac1N\sum_{t=1}^N X_t-\expt{X}\right|\ge \epsilon\right) &=0.
\end{align*}
\end{theorem}
\begin{proof}
チェビシェフの不等式より
\begin{align*}
\Pr\left(\left|\frac1N\sum_{t=1}^N X_t-\expt{X}\right|\ge \epsilon\right) &\le
\frac{\var{\frac1N\sum_{t=1}^NX_t}}{\epsilon} = \frac{\var{X}}{\epsilon^2\emm{N}}\stackrel{N\to\infty}{\longrightarrow} 0.
\end{align*}
\end{proof}
\end{frame}

\begin{frame}{モーメント母関数}
\begin{definition}
確率変数$X$と実数$t$について
\begin{align*}
M_X(t) &= \expt{\exp(tX)}
\end{align*}
を\emm{モーメント母関数}(積率母関数)という。
\end{definition}
\begin{align*}
M_X(t) &= \expt{\exp(tX)} = \expt{X} t + \frac{\expt{X^2}}2 t^2 + \frac{\expt{X^3}}{3!} t^3 + \dotsb
\end{align*}
\begin{example}
\end{example}
\end{frame}

\begin{frame}{チェルノフ上界}
\begin{theorem}
任意の実数$t>0$について
\begin{align*}
\Pr(X\ge a)&\le\frac{M_X(t)}{\exp(at)}.
\end{align*}
\end{theorem}
\begin{proof}
\begin{align*}
\Pr(X\ge a)&\le\Pr\bigl(\exp(tX)\ge\exp(ta)\bigr)\\
&\le\frac{\expt{\exp(tX)}}{\exp(ta)}\\
&\le\frac{M_X(t)}{\exp(at)}
\end{align*}
\end{proof}
\end{frame}

\begin{frame}{チェルノフ上界の最適化}
任意の実数$t>0$について
\begin{align*}
\Pr(X\ge a)&\le\frac{M_X(t)}{\exp(at)}
\end{align*}
が成り立つので上界の最小化
\begin{align*}
\inf_{t>0} \frac{M_X(t)}{\exp(at)}
\end{align*}
を考えたい。$\log$とってから最小化すると
\begin{align*}
\exp\left(\inf_{t>0} \{\emm{\log M_X(t)} - at\}\right)
&=
\exp\left(-\sup_{t>0} \{at - \emm{\log M_X(t)}\}\right)
.
\end{align*}
\emm{キュムラント母関数}($\log M_X(t)$)の\emm{ルジャンドル変換}
\end{frame}

\begin{frame}{ルジャンドル変換}
実関数$f\colon\mathbb{R}\to\mathbb{R}$について
\begin{align*}
f^*(y) &= \sup_x\{yx - f(x)\}
\end{align*}
\end{frame}

\begin{frame}{集中不等式 by チェルノフ上界}
\begin{lemma}
\begin{align*}
\Pr\left(\frac1N\sum_{t=1}^N X_t-\expt{X}\ge a\right) &\le \left(\frac{M_X(t)}{\exp\left(t(\expt{X}+ a)\right)}\right)^N\qedhere
\end{align*}
\end{lemma}
\begin{proof}
\small
\begin{align*}
\Pr\left(\frac1N\sum_{t=1}^N X_t\ge \expt{X}+ a\right) &=
%\Pr\left(\exp\left(t\sum_{t=1}^N X_t\right)\ge \exp\left(t(\expt{X} + a)N\right)\right)\\
\Pr\left(\sum_{t=1}^N X_t\ge N(\expt{X}+ a)\right)\\
&\le\frac{M_X(t)^N}{\exp\left(t(\expt{X}+ a)N\right)}\\
&=\left(\frac{M_X(t)}{\exp\left(t(\expt{X}+ a)\right)}\right)^N.\qedhere
\end{align*}
\end{proof}
\end{frame}

\begin{frame}{大偏差原理 Lv1}
\begin{theorem}[クラメールの定理]
\end{theorem}
\end{frame}

\begin{frame}{大偏差原理 Lv2}
\begin{theorem}[サノフの定理]
\end{theorem}
\end{frame}
\end{document}
