\documentclass[lualatex,handout]{beamer}
\setbeamertemplate{footline}[frame number]
\usepackage{luatexja}
\usepackage{amsmath,amssymb}

%\usetheme{Berlin}
\usecolortheme{rose}

\usepackage{tikz}
\usepackage{pgfplots}
\pgfplotsset{compat=1.18}

%\usepackage[haranoaji]{luatexja-preset}
\usepackage[deluxe,ipaex]{luatexja-preset}
\renewcommand{\kanjifamilydefault}{\gtdefault}
%\setmainjfont{HaranoAjiGothic-Regular}

\usepackage{unicode-math}
%\setmathfont{Fira Math}
\setmathfont{STIX Two Math}
\setmathfont{STIX Two Math}[range=bfup/{Latin,latin,num,Greek,greek}]
\setmathfont{STIX Two Math}[range=bfit/{Latin,latin}]
\setmathfont{STIX Two Math}[range={"0000-"FFFF}]
\setmathrm{STIX Two Math}[StylisticSet=8]

%\usefonttheme{professionalfonts}

\usepackage{luacolor}

\newcommand{\mycolor}[2]{%
  \begingroup
  \colorlet{currentcolor}{.}%
  \color{#1}#2%
  \color{currentcolor}%
  \endgroup
}
\newcommand{\emm}[1]{\mycolor{red}{#1}}
\newcommand{\expt}[1]{\mathbb{E}\left[#1\right]}
\newcommand{\var}[1]{\mathbb{V}\left[#1\right]}
\newcommand{\cov}[1]{\mathsf{Cov}\left[#1\right]}
\newcommand{\vc}[1]{\mathsf{Var}\left[#1\right]}


\usepackage{xspace}
%\usepackage{bm}
%\newcommand\bm[1]{{\mathbf{#1}}}
\newcommand\dx{{\,\mathrm{d}x}}

\theoremstyle{definition}

\title{確率・統計基礎: サノフの定理、エントロピー、KLダイバージェンス、}
\author{森 立平}
\date{}



\begin{document}
\begin{frame}[plain]
\maketitle
\end{frame}

\begin{frame}{経験分布の集中}
大数の法則や中心極限定理と同様にi.i.d.確率変数$X_1,X_2,\dotsc,X_N$を考える。

\vspace{1em}
%\begin{definition}
確率変数$X$の像が\emm{有限集合}$A$とする。

各$a\in A$について$N_a := \{ k\mid X_k = a\}$とおく。

$\left(\frac{N_a}{N}\right)_{a\in A}$を経験分布という。
%\end{definition}

\vspace{1em}
サイコロを100回振って1,2,3,4,5,6がそれぞれ16,13,18,20,14,19回出たとき経験分布は
\begin{align*}
P(1) &= \frac{16}{100}&
P(2) &= \frac{13}{100}&
P(3) &= \frac{18}{100}\\
P(4) &= \frac{20}{100}&
P(5) &= \frac{14}{100}&
P(6) &= \frac{19}{100}
\end{align*}
である。
\end{frame}

\begin{frame}{経験分布}
\begin{definition}
確率変数$X$の像が\emm{有限集合}$A$とする。

各$a\in A$について$N_a := \{ k\mid X_k = a\}$とおく。

$\left(\frac{N_a}{N}\right)_{a\in A}$を経験分布という。
\end{definition}
\end{frame}

\begin{frame}{サノフの定理}
\end{frame}

\begin{frame}{文字列のタイプ}

\end{frame}

\begin{frame}{確率}
\begin{align*}
\sum_{}\prod_{k=1}^N \Pr(X = x_k) &= \binom{N}{N_1, N_2, \dotsc, N_k} \prod_{k=1}^N\Pr(X=a_k)^{q(k)N}\\
&= \mathrm{e}^{-\sum_kq(k)\log q(k) N + (\sum_{k} q(k)\log p(k))N}\\
&= \mathrm{e}^{-\left(\sum_kq(k)\log \frac{q(k)}{p(k)}\right) N}\\
\end{align*}

\begin{definition}[Kullback--Libler divergence]
\begin{align*}
D(q\,\|\,p) &:= -\sum_kq(k)\log \frac{q(k)}{p(k)}
\end{align*}
\end{definition}
\end{frame}

\if0
\begin{frame}{$n!$の上下界}
\footnotesize
\begin{lemma}
任意の自然数$n$について
\begin{align*}
\left(\frac{n}{\mathrm{e}}\right)^n\le n!\le n^n.
\end{align*}
\vspace{-1em}
\begin{proof}
$n!\le n^n$は自明である。以下では$\frac1{n!}\le \left(\frac{\mathrm{e}}{n}\right)^n$を示す。
%
$C_r$を原点を中心とした半径$r$の円を反時計回りに回るパスとすると、任意の$r>0$について
\begin{align*}
\frac1{n!} &= \frac1{2\pi i} \oint_{C_r} \frac{\mathrm{e}^z}{z^{n+1}}\mathrm{d}z\\
 &= \left|\frac1{2\pi i} \oint_{C_r} \frac{\mathrm{e}^z}{z^{n+1}}\mathrm{d}z\right|\\
&\le \frac1{2\pi } \oint_{C_r} \left|\frac{\mathrm{e}^z}{z^{n+1}}\right|\mathrm{d}z\\
&\le \frac1{2\pi } (2\pi r) \frac{\mathrm{e}^r}{r^{n+1}} = \frac{\mathrm{e}^r}{r^n}
\end{align*}
である。この上界は$r=n$のときに最小化されて$\frac1{n!}\le\left(\frac{\mathrm{e}}{n}\right)^n$を得る。
\end{proof}
\end{lemma}
\end{frame}
\fi

\begin{frame}{多項係数の上下界}
\begin{lemma}
\begin{align*}
\frac1{(N+1)^m}\mathrm{e}^{N H(p)}\le
\binom{N}{N_1\,N_2\,\dotsm N_m}
\le \mathrm{e}^{N H(p)}
\end{align*}
\end{lemma}
\begin{proof}
任意の$N_1$と$p \in\mathcal{P}_m$について
\begin{align*}
1&\ge \binom{N}{N_1\,N_2\,\dotsm N_m} \prod_{k=1}^m q_k^{N_k}
\end{align*}
よって$q_k=p_k$とすると
\begin{align*}
\binom{N}{N_1\,N_2\,\dotsm N_k}&\le \frac{1}{\prod_{k=1}^m \left(\frac{N_k}{N}\right)^{N_k}} = \mathrm{e}^{NH(p)}
\end{align*}
\end{proof}
\end{frame}


\end{document}
