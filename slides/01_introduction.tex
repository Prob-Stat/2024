\documentclass[lualatex,handout]{beamer}
\usepackage{luatexja}
\usepackage{amsmath,amssymb}

\usepackage{tikz}

%\usepackage[haranoaji]{luatexja-preset}
\usepackage[ipaex]{luatexja-preset}
\renewcommand{\kanjifamilydefault}{\gtdefault}
%\setmainjfont{HaranoAjiGothic-Regular}

\usepackage{unicode-math}
%\setmathfont{Fira Math}
%\setmathfont{STIX Two Math}
%\setmathfont{STIX Two Math}[range={up,it,bb,bbit,scr,bfscr},StylisticSet=8]
\setmathfont{XITS Math}
%\setmathfont{XITS Math}[StylisticSet=8]
\setmathrm{XITS Math}[StylisticSet=8]

%\usefonttheme{professionalfonts}

\usepackage{luacolor}

\newcommand{\emm}[1]{\textcolor{red}{#1}}
\newcommand{\expt}[1]{\mathbb{E}[#1]}
\newcommand{\var}[1]{\mathbb{V}[#1]}
\newcommand{\cov}[1]{\mathrm{Cov}[#1]}


\usepackage{xspace}
\newcommand\bm[1]{{\mathbf{#1}}}

\theoremstyle{definition}

\title{確率・統計基礎: 確率論の初歩のおさらい}
\author{森 立平}
\date{}



\begin{document}
\begin{frame}[plain]
\maketitle
\end{frame}


\begin{frame}{確率とは}
\begin{itemize}
\setlength{\itemsep}{2em}
\item 世の中のランダムな事象を数学的に記述したもの
\item 統計・機械学習の基礎となっている
\item 極限定理: 確率$p$で表が出るコインを$N$回投げた時に、表が出る回数$F$は
\vspace{1em}
\begin{itemize}
\setlength{\itemsep}{1em}
\item 大数の法則: $\frac{F}{N} \to p$
\item 大偏差原理: $\Pr\left(\left|\frac{F}{N} - p\right| > \epsilon\right) = \mathrm{e}^{-\beta N}$
\item 中心極限定理: $\frac{F-pN}{\sqrt{N}}\to$正規分布
\end{itemize}
\end{itemize}
\end{frame}

\begin{frame}{統計とは}
確率論は統計の基礎となる
\vspace{1em}
\begin{itemize}
\item クラスタリング: この患者は病気ですか?
\item a
\end{itemize}
\end{frame}

\begin{frame}{確率論の数学モデルとは?}
\begin{center}
\large 測度論!
\end{center}

\vspace{1em}
\begin{itemize}
\setlength{\itemsep}{2em}
\item 確率は面積($=$測度)のようなもの。
\item 数学的に\textbf{厳密}に様々な結果(大数の法則、大偏差原理、中心極限定理)が証明できる。
\item とても重要だが勉強するのは3年生以降(解析学要論Ⅱ、確率論)。
\item この授業では測度論を使わない代わりに制限された確率論をやる。
\end{itemize}
\end{frame}

\begin{frame}{離散的な確率分布}
\begin{itemize}
\item $p_1, p_2, \dotsc, p_n\ge 0$, $p_1+p_2+\dotsb+p_n=0$.
\item $\mathrm{abc}$
\end{itemize}
\end{frame}

\begin{frame}{確率変数}
$X\colon $
\begin{align*}
\Pr(X\ge a)
\end{align*}
\end{frame}

\begin{frame}{同時確率と条件付き確率}

\end{frame}

\begin{frame}{独立確率変数}

\end{frame}

\begin{frame}{平均}
\begin{align*}
\expt{X} &= \sum_i p_i x_i
\end{align*}
線形
\begin{align*}
\expt{aX + bY} &= a\expt{X} + b\expt{Y} 
\end{align*}
\end{frame}

\begin{frame}{マルコフの不等式}
\begin{theorem}[マルコフの不等式]
任意の\emm{非負}の確率変数$X$と$a>0$について
\begin{align*}
\Pr(X\ge a) &\le \frac{\expt{X}}{a}
\end{align*}
\end{theorem}
\begin{proof}
\vspace{-2em}
\begin{align*}
\expt{X} &= \sum_i p_i x_i\\
&= \sum_{i\colon x_i \ge a} p_i x_i
+ \sum_{i\colon x_i < a} p_i x_i\\
&\ge \sum_{i\colon x_i \ge a} p_i x_i\hspace{8em}(x_i\ge 0)\\
&\ge \sum_{i\colon x_i \ge a} p_i a\\
&= a\Pr(X\ge a) \qedhere
\end{align*}
\end{proof}
\end{frame}

\begin{frame}{例題}
\end{frame}

\begin{frame}{分散}
\begin{align*}
\var{X} &= \expt{(X-\expt{X})^2}\\
&=\expt{X^2-2X\expt{X}+\expt{X}^2}\\
&=\expt{X^2}-2\expt{X}\expt{X}+\expt{X}^2\\
&=\emm{\expt{X^2}-\expt{X}^2}\\
\end{align*}
\begin{center}
分散は\emm{非負}で\emm{平均からの広がり}を表す量。
\end{center}
\end{frame}

\begin{frame}{分散の意味}
\begin{align*}
\var{X} &= \expt{(X-\expt{X})^2}
\end{align*}

$\var{X} = 0 \iff \Pr(X = \expt{X})=1$.
\end{frame}

\begin{frame}{分散の性質}
\begin{align*}
\var{X} &= \expt{(X-\expt{X})^2}
\end{align*}
%\vspace{1em}
\begin{align*}
\var{aX} &= \expt{(aX-\expt{aX})^2}\\
&= \expt{(aX-a\expt{X})^2}\\
&= \expt{a^2(X-\expt{X})^2}\\
&= a^2\expt{(X-\expt{X})^2}\\
&= a^2\var{X}
\end{align*}
%
\begin{align*}
\var{X+Y} &= \expt{(X+Y-\expt{X+Y})^2}\\
 &= \expt{(X-\expt{X}+Y-\expt{Y})^2}\\
 &= \expt{(X-\expt{X})^2+(Y-\expt{Y})^2 + 2(X-\expt{X})(Y-\expt{Y})}\\
 &= \expt{(X-\expt{X})^2}+\expt{(Y-\expt{Y})^2} + 2\expt{(X-\expt{X})(Y-\expt{Y})}\\
 &= \var{X}+\var{Y}+ 2\emm{\expt{(X-\expt{X})(Y-\expt{Y})}}\\
\end{align*}
\end{frame}

\begin{frame}{共分散}
\begin{align*}
\cov{X,Y} &= \expt{(X-\expt{X})(Y-\expt{Y})}
\end{align*}

\vspace{1em}
直感的な意味
\end{frame}

\begin{frame}{無相関と独立}
\begin{align*}
\cov{X,Y} &= \expt{(X-\expt{X})(Y-\expt{Y})} = 0
\end{align*}
\end{frame}

\begin{frame}{チェビシェフの不等式}
\begin{lemma}[チェビシェフの不等式]
任意の $a>0$について
\begin{align*}
\Pr(|X-\expt{X}|\ge a) &\le \frac{\var{X}}{a^2}
\end{align*}
\end{lemma}
\begin{proof}
\begin{align*}
\Pr(|X-\expt{X}|\ge a) &=\Pr((X-\expt{X})^2\ge a^2)\\
&\le\frac{\expt{(X-\expt{X})^2}}{a^2}\hspace{3em}\text{(チェビシェフの不等式)}
\end{align*}
\end{proof}
\end{frame}

\end{document}
