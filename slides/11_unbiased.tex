\documentclass[lualatex,handout]{beamer}
\setbeamertemplate{footline}[frame number]
\usepackage{luatexja}
\usepackage{amsmath,amssymb}

\usepackage{thm-restate}

%\usetheme{Berlin}
\usecolortheme{rose}

\usepackage{tikz}
\usepackage{pgfplots}
\pgfplotsset{compat=1.18}

%\usepackage[haranoaji]{luatexja-preset}
\usepackage[deluxe,ipaex]{luatexja-preset}
\renewcommand{\kanjifamilydefault}{\gtdefault}
%\setmainjfont{HaranoAjiGothic-Regular}

\usepackage{unicode-math}
%\setmathfont{Fira Math}
\setmathfont{STIX Two Math}
\setmathfont{STIX Two Math}[range=bfup/{Latin,latin,num,Greek,greek}]
\setmathfont{STIX Two Math}[range=bfit/{Latin,latin}]
\setmathfont{STIX Two Math}[range={"0000-"FFFF}]
\setmathrm{STIX Two Math}[StylisticSet=8]

%\usefonttheme{professionalfonts}

\usepackage{luacolor}

\newcommand{\mycolor}[2]{%
  \begingroup
  \colorlet{currentcolor}{.}%
  \color{#1}#2%
  \color{currentcolor}%
  \endgroup
}
\newcommand{\emm}[1]{\mycolor{red}{#1}}
\newcommand{\expt}[1]{\mathbb{E}\left[#1\right]}
\newcommand{\var}[1]{\mathbb{V}\left[#1\right]}
\newcommand{\cov}[1]{\mathsf{Cov}\left[#1\right]}
\newcommand{\vc}[1]{\mathsf{Var}\left[#1\right]}

\DeclareMathOperator*{\argmax}{arg\,max}


\usepackage{xspace}
%\usepackage{bm}
%\newcommand\bm[1]{{\mathbf{#1}}}
\newcommand\defiff{\stackrel{\text{def}}{\iff}}
\newcommand\dx{{\,\mathrm{d}x}}
\newcommand\KL[2]{D\left(#1\,\|\,#2\right)}
\newcommand\dtv{d_{\mathrm{TV}}}

\theoremstyle{definition}
%\newtheorem{fact}{Fact}

\title{確率・統計基礎: 点推定、不偏推定量}
\author{森 立平}
\date{}



\begin{document}
\begin{frame}[plain]
\maketitle
\end{frame}


\begin{frame}{二乗誤差}
\small
\begin{align*}
\mathrm{MSE}(\theta,\,\delta)&\coloneq \expt{(\delta(X)-g(\theta))^2\mid\theta}\\
&= \expt{\left(\delta(X)\emm{-\expt{\delta(X)\mid\theta}+\expt{\delta(X)\mid\theta}}-g(\theta)\right)^2\mid\theta}\\
&= \expt{\left(\delta(X)-\expt{\delta(X)\mid\theta}\right)^2\mid\theta}+\expt{\left(\expt{\delta(X)\mid\theta}-g(\theta)\right)^2\mid\theta}\\
&\quad + 2\expt{\left(\delta(X)-\expt{\delta(X)\mid\theta}\right)\left(\expt{\delta(X)\mid\theta}-g(\theta)\right)\mid\theta}\\
&= \var{\delta(X)\mid\theta}+\left(\expt{\delta(X)\mid\theta}-g(\theta)\right)^2\\
&= \var{\delta(X)\mid\theta}+\mathrm{Bias}_g(\delta\mid\theta)^2
%\expt{\widehat{\theta}(X)^2 - 2\widehat{\theta}(X)\theta + \theta^2}
\end{align*}
\begin{align*}
\mathrm{Bias}_g(\delta\mid\theta)&\coloneq\expt{\delta(X)\mid\theta}-g(\theta)
\end{align*}
\begin{definition}[不偏推定量]
推定量$\delta\colon \mathcal{X}\to\mathbb{R}$が以下を満たすとき、$g(\theta)$の\emm{不偏推定量}であるという。
\begin{align*}
\mathrm{Bias}_g(\delta\mid\theta) &= 0 \qquad\forall \theta\in\Theta.
\end{align*}
\end{definition}
\end{frame}

\begin{frame}{不偏推定量は存在するとは限らない}
パラメータ$\theta\in(0,1)$について、
$X\sim\mathrm{Binom}(N, \theta)$、$g(\theta) = 1/\theta$とする。
$\delta\colon \mathbb{Z}_{\ge 0}\to \mathbb{R}$を不偏推定量とすると、
\begin{align*}
\frac1\theta &= \sum_{x=0}^N \delta(x) \Pr(X = x)\\
&= \sum_{x=0}^N \delta(x) \binom{N}{x} \theta^x(1-\theta)^{N-x}
\end{align*}
$\theta\to 0$の極限で左辺は$+\infty$に発散する。一方で右辺は$\delta(0)$に収束する。
よって\emm{$\delta$をどのように定めても不偏推定量にはならない}。
\end{frame}

\if0
\begin{frame}{ジャックナイフサンプリング}
\begin{align*}
\Pr(\symbf{X}=\symbf{x}\mid\theta) &= \prod_{k=1}^N \Pr(X_k=x_k\mid\theta)
\end{align*}
各$N\in\mathbb{N}$について
$\delta_N\colon \mathcal{X}^N\to\mathbb{R}$が
\begin{align*}
\expt{\delta_N(\symbf{X})\mid\theta} &= g(\theta) + O\left(\frac{1}{N}\right)
\end{align*}
を満たすとする。
\end{frame}
\fi

\begin{frame}{不偏推定量の自由度}
\begin{fact}
$\delta_0\colon\mathcal{X}\to\mathbb{R}$を$g(\theta)$の不偏推定量とする。
任意の推定量$\delta\colon\mathcal{X}\to\mathbb{R}$について以下は同値
\begin{enumerate}
\item $\delta$は$g(\theta)$の不偏推定量
\item $\delta-\delta_0$は $0$の不偏推定量
\end{enumerate}
\end{fact}
よって$h\colon\mathcal{X}\to\mathbb{R}$を$0$の不偏推定量とすると、$g(\theta)$の任意の不偏推定量は$\delta=\delta_0+h$と表せる。このとき、
\begin{align*}
\var{\delta(X)\mid\theta}&=
\var{\delta_0(X)+h(X)\mid\theta}\\
&= \var{\delta_0(X)\mid\theta}+\var{h(X)\mid\theta}+2\cov{\delta_0(X), h(X)\mid\theta}\\
&= \var{\delta_0(X)\mid\theta}+\emm{\expt{h(X)^2\mid\theta}+2\expt{\delta_0(X)h(X)\mid\theta}}
%&=
%\expt{(\delta_0(X)+h(X))^2\mid\theta}-
%\expt{\delta_0(X)+h(X)\mid\theta}^2\\
%&=
%\emm{\expt{(\delta_0(X)+h(X))^2\mid\theta}}- g(\theta)^2
\end{align*}
%よって$\emm{\expt{(\delta_0(X)+h(X))^2\mid\theta}}$を最小化する$h$を見つければよい。
よって$\emm{\expt{h(X)^2\mid\theta}+2\expt{\delta_0(X)h(X)\mid\theta}}$を最小化する$h$を見つければよい。
\end{frame}

\begin{frame}{例: 不偏推定量の最適化 1/2}
\small
パラメータ$p\in(0,1)$について、
\begin{align*}
\Pr(X=-1\mid p) &= p\\
\Pr(X=x\mid p) &= p^x (1-p)^2 \qquad\forall x\in\mathbb{Z}_{\ge 0}
\end{align*}
とする(確率$p$で$-1$、確率$1-p$で幾何分布$\mathrm{Geo}(p)$)。このとき、$\emm{p}$と$\emm{(1-p)^2}$の不偏推定量をそれぞれ
\begin{align*}
\delta_0(x) &=\mathbb{1}_{\{x = -1\}}\\
\delta_1(x) &=\mathbb{1}_{\{x = 0\}}
\end{align*}
とする。
また、$h\colon\mathbb{Z}_{\ge -1}\to\mathbb{R}$を$0$の不偏推定量とすると、
\begin{align*}
&ph(-1) + \sum_{x\ge 0} h(x) p^x (1-p)^2 = 0\qquad\forall p\in(0,1)\\
%&h(0) + (h(-1) + h(1) - 2h(0))p + \sum_{x\ge 2} (h(x)-2h(x-1)+h(x-2)) p^x = 0\qquad\forall p\in(0,1)\\
\iff&h(0) + \sum_{x\ge 1} (h(x)-2h(x-1)+h(x-2)) p^x = 0\qquad\forall p\in(0,1)\\
\iff&h(0)=0,\qquad h(x)-2h(x-1)+h(x-2) = 0\qquad\forall x\in\mathbb{Z}_{\ge1}\\
\iff&h(x)=xh(1)\qquad\forall x\in\mathbb{Z}_{\ge-1}\\
\end{align*}
\end{frame}

\begin{frame}{例: 不偏推定量の最適化 2/2}
\small
%\begin{align*}
% \emm{\expt{(\delta_0(X)+h(X))^2\mid p}}
%&= \expt{(\mathbb{1}_{\{X=-1\}}+Xh(1))^2\mid p}\\
%&= \expt{\mathbb{1}_{\{X=-1\}}+2\mathbb{1}_{\{X=-1\}}Xh(1) + X^2h(1)^2\mid p}\\
%&= p -2ph(1) + \left(p + (1-p)\frac{p+p^2}{(1-p)^2}\right)h(1)^2\\
%&= p -2ph(1) + \frac{2p}{1-p}h(1)^2
%\end{align*}
%よって$h(1) = \frac{1-p}2$のとき最小化される。最適な不偏推定量は$p$毎に違う。
\begin{align*}
&\emm{\expt{h(X)^2\mid\theta}+2\expt{\delta_0(X) h(X)\mid\theta}}\\
&=\expt{(\emm{h(1)}X)^2\mid\theta} + 2\expt{\delta_0(X) \emm{h(1)}X\mid\theta}\\
&=\expt{X^2\mid\theta}\emm{h(1)}^2 + 2\expt{\delta_0(X) X\mid\theta}\emm{h(1)}
%&=\expt{(Xh(1))^2\mid\theta} + 2\expt{(\mathbb{1}_{\{X=-1\}}-p)Xh(1)\mid\theta}\\
%&=\expt{X^2\mid\theta}h(1)^2 + 2\expt{(\mathbb{1}_{\{X=-1\}}-p)X\mid\theta}h(1)\\
\end{align*}
よって$\emm{h(1)} = -\frac{\expt{\delta_0(X)X\mid\theta}}{\expt{X^2\mid\theta}}=\frac{1-p}2$のとき最小化される。
最適な不偏推定量は$p$毎に違う。

\vspace{2em}
%\begin{align*}
 %\emm{\expt{(\delta_1(X)+h(X))^2\mid p}}
%&= \expt{(\mathbb{1}_{\{X=0\}}+Xh(1))^2\mid p}\\
%&= \expt{\mathbb{1}_{\{X=0\}}+2\mathbb{1}_{\{X=0\}}Xh(1) + X^2h(1)^2\mid p}\\
%&= (1-p)^2 + \frac{2p}{1-p}h(1)^2
%\end{align*}
%よって$h(1) = 0$のとき最小化される。\emm{最適な不偏推定量は$p$によらず}$\delta_1(x)=\mathbb{1}_{\{x=0\}}$.
同様に$\delta_1$について考えると$\emm{h(1)} = -\frac{\expt{\delta_1(X)X\mid\theta}}{\expt{X^2\mid\theta}}=0$のとき最小化される。
\emm{最適な不偏推定量は$p$によらず}$\delta_1(x)=\mathbb{1}_{\{x=0\}}$.
\end{frame}

\begin{frame}{一様最小分散不偏推定量}
\small
\begin{definition}[一様最小分散不偏推定量 (uniform minimum variance unbiased (UMVU) estimator)]
不偏推定量$\delta\colon\mathcal{X}\to\mathbb{R}$が任意の$\theta\in\Theta$について、最小の分散を持つとき\emm{一様最小分散不偏推定量}という。
\begin{enumerate}
\item (不偏推定量である) $\expt{\delta(X)\mid\theta}=g(\theta)$ for all $\theta\in\Theta$.
\item (分散が最小である) $\var{\delta(X)\mid\theta}\le \var{\delta'(X)\mid\theta}$が任意の不偏推定量$\delta'\colon\mathcal{X}\to\mathbb{R}$と$\theta\in\Theta$について成り立つ。
\end{enumerate}
\end{definition}
\begin{align*}
\var{\delta(X)+ah(X)\mid\theta}
&= \var{\delta(X)\mid\theta}+\var{ah(X)\mid\theta}+2\cov{\delta(X), ah(X)\mid\theta}\\
&= \var{\delta(X)\mid\theta}+a^2\var{h(X)\mid\theta}+2a\cov{\delta(X), h(X)\mid\theta}
\end{align*}
よって$\delta$がUMVU推定量のとき、0の任意の不偏推定量$h$について$\cov{\delta(X), h(X)\mid\theta}=0$でなければならない。
\end{frame}

\begin{frame}{一様最小分散不偏推定量}
\small
\begin{theorem}

\vspace{-2em}
\begin{align*}
\Delta&\coloneq\{\delta\colon\mathcal{X}\to\mathbb{R}\mid \expt{\delta(X)^2\mid\theta}<\infty \quad\forall\theta\in\Theta\}\\
\mathcal{U}&\coloneq\{h\colon\mathcal{X}\to\mathbb{R}\mid \expt{h(X)\mid\theta}=0 \quad\forall\theta\in\Theta\} \cap\Delta
\end{align*}
とする。$\delta\in\Delta$について以下は同値
\begin{enumerate}
\item $\delta$はUMVU推定量。
\item 任意の$h\in\mathcal{U}$と$\theta\in\Theta$について$\emm{\expt{\delta(X)h(X)\mid\theta}=0}$。
\end{enumerate}
\end{theorem}
\begin{proof}
$1\implies 2$:
$\delta$はUMVU推定量と仮定する。
任意の$h\in\mathcal{U}$と$\theta\in\Theta$と$a\in\mathbb{R}$について
\begin{align*}
\var{\delta(X)\mid\theta}&\le\var{\delta(X)+ah(X)\mid\theta}\\
 &= \var{\delta(X)\mid\theta}+\var{h(X)\mid\theta}a^2+2\expt{\delta(X)h(X)\mid\theta}a
\end{align*}
が成り立つので
%($h(x)=0$のときとそうでないときで場合分けすると)
、$\expt{\delta(X)h(X)\mid\theta}=0$。
\end{proof}
\end{frame}

\begin{frame}{一様最小分散不偏推定量}
\small
\begin{theorem}

\vspace{-2em}
\begin{align*}
\Delta&\coloneq\{\delta\colon\mathcal{X}\to\mathbb{R}\mid \expt{\delta(X)^2\mid\theta}<\infty \quad\forall\theta\in\Theta\}\\
\mathcal{U}&\coloneq\{h\colon\mathcal{X}\to\mathbb{R}\mid \expt{h(X)\mid\theta}=0 \quad\forall\theta\in\Theta\} \cap\Delta
\end{align*}
とする。$\delta\in\Delta$について以下は同値
\begin{enumerate}
\item $\delta$はUMVU推定量。
\item 任意の$h\in\mathcal{U}$と$\theta\in\Theta$について$\emm{\expt{\delta(X)h(X)\mid\theta}=0}$。
\end{enumerate}
\end{theorem}
\begin{proof}
$2\implies 1$:
任意の$h\in\mathcal{U}$と$\theta\in\Theta$について$\expt{\delta(X)h(X)\mid\theta}=0$ と仮定する。
$\delta'\colon\mathcal{X}\to\mathbb{R}$を不偏推定量とする。$\var{\delta'(X)\mid\theta}=\infty$の場合は
$\var{\delta(X)\mid\theta}\le\var{\delta'(X)\mid\theta}$である。
$\delta'\in\Delta$の場合は$h\coloneq\delta'-\delta\in\mathcal{U}$。
このとき任意の$\theta\in\Theta$について
\begin{align*}
\var{\delta'(X)\mid\theta} &= \var{\delta(X)+h(X)\mid\theta}\\
&= \var{\delta(X)}+\expt{h(X)^2\mid\theta} + 2\expt{\delta(X)h(X)\mid\theta}\\
&= \var{\delta(X)}+\expt{h(X)^2\mid\theta} \ge \var{\delta(X)}
\end{align*}
\end{proof}
\end{frame}

\begin{frame}{推定量の平均}
\footnotesize
\begin{lemma}
$\widehat{\theta}_1,\,\widehat{\theta}_2,\dotsc,\widehat{\theta}_n\colon A\to\Theta$が不偏推定量のとき、
\begin{align*}
\widehat{\theta}^* &\coloneq \frac1n\sum_{k=1}^n\widehat{\theta}_k
\end{align*}
は不偏推定量で
\begin{align*}
\var{\widehat{\theta}^*(X)\mid\theta} &\le\frac1n\sum_{k=1}^n\var{\widehat{\theta}_k(X)\mid\theta}
\qquad\forall\theta\in\Theta.
\end{align*}
\end{lemma}
\begin{proof}
$\widehat{\theta}^*$が不偏推定量であることは期待値の線形性から明らか。
\begin{align*}
\var{\widehat{\theta}^*(X)\mid\theta} &=
\expt{\left(\frac1n\sum_{k=1}^n\widehat{\theta}_k(X) - \theta\right)^2\mid\theta}\\
&\le
\frac1n\sum_{k=1}^n\expt{\left(\widehat{\theta}_k(X) - \theta\right)^2\mid\theta}\quad\text{(イェンセンの不等式)}\qedhere
%\frac1{n^2} \left(\sum_{k=1}^n \var{\widehat{\theta}_k(X)\mid\theta} + 2\sum_{k<\ell} \cov{\widehat{\theta}_k,\,\widehat{\theta}_\ell}\right)
\end{align*}
\end{proof}
\end{frame}

\begin{frame}{対称な不偏推定量}
\small
\begin{align*}
p(\symbf{x}\mid \theta) &= \prod_{k=1}^N p(x_k\mid \theta).
\end{align*}
\begin{lemma}[対称な不偏推定量]
任意の不偏推定量$\widehat{\theta}\colon A^N\to\Theta$について、ある対称な不偏推定量$\widehat{\theta}^*$が存在して
\begin{align*}
\var{\widehat{\theta}^*(\symbf{X})\mid\theta} &\le \var{\widehat{\theta}(\symbf{X})\mid\theta}
\qquad\forall\theta\in\Theta.
\end{align*}
\end{lemma}
\begin{proof}
\begin{align*}
\widehat{\theta}^*(x_1,\dotsc,x_N) &\coloneq \frac1{N!} \sum_{\sigma\in S_N} \widehat{\theta}\left(x_{\sigma(1)},\dotsc,x_{\sigma(N)}\right)
\end{align*}
と定義すればよい。
\end{proof}
\end{frame}

\begin{frame}{加法的ノイズのモデル}
\footnotesize
\begin{align*}
X &= \theta + Z
\end{align*}
\begin{itemize}
\item $\expt{Z} = 0$.
\item $\var{Z} < \infty$.
\end{itemize}
\begin{theorem}
$\widehat{\theta}^*$が$Z$の分布によらず対称な不偏推定量となるとき、$\widehat{\theta}^*$は\emm{標本平均}
\begin{align*}
\widehat{\theta}^*(x_1,\dotsc,x_n)&\coloneq \frac1n \sum_{k=1}^n x_k
\end{align*}
である。
\end{theorem}
\begin{proof}
$n=1$のとき、
$\Pr(Z=0)=1$と仮定すると、
\begin{align*}
\theta&= \expt{\widehat{\theta}^*(X)\mid\theta}
= \widehat{\theta}^*(\theta)
\end{align*}
が任意の$\theta\in\Theta$について成り立つ。よって$\widehat{\theta}^*(x)=x$である。
\end{proof}
\end{frame}

\begin{frame}{加法的ノイズのモデル}
\small
ある$a\ne 0$と$p\in[0,1]$について、ノイズの分布を
\begin{align*}
\Pr(Z = a(p-1)) &= p&
\Pr(Z = ap) &= 1-p
\end{align*}
とすると$\expt{Z}=0$, $\var{Z}<\infty$である。

$n=2$のとき、
任意の$a\ne 0$と$p\in[0,1]$について
\begin{align*}
%\expt{\widehat{\theta}^*(\symbf{X})\mid\theta} &=
\theta&= \expt{\widehat{\theta}^*(X_1,X_2)\mid\theta}\\
 &= p^2 \widehat{\theta}^*(a(p-1),a(p-1)) + (1-p)^2 \widehat{\theta}^*(ap, ap)\\
&\quad + p(1-p) \widehat{\theta}^*(a(p-1), ap) + (1-p)p\widehat{\theta}^*(ap, a(p-1))\\
 &= p^2 a(p-1) + (1-p)^2 ap + 2p(1-p) \widehat{\theta}^*(ap, a(p-1))
\end{align*}
が成り立つ。
\end{frame}

\begin{frame}{課題}
\small
以下の問に答えよ。
\begin{enumerate}
\setlength{\itemsep}{1em}
\item 
\begin{align*}
p(\symbf{x}\mid\theta)&= \frac1{\sqrt{(2\pi)^N}}\mathrm{e}^{-\sum_{k=1}^N\frac{(x_k-\theta)^2}2}
\end{align*}
のとき、以下の$T\colon\mathbb{R}^N\to\mathbb{R}$が十分統計量かどうか示せ。結論だけでなく理由も示すこと。
\begin{align*}
T(\symbf{x})&= \sum_{k=1}^N x_k.
\end{align*}
\item
\begin{align*}
\Pr(X_1=x_1,\dotsc,X_N=x_N\mid\lambda)&= \frac1{\prod_{k=1}^N x_k!}\lambda^{\sum_{k=1}^N x_k} \mathrm{e}^{-N\lambda}
\end{align*}
のとき、$\lambda$の推定関数$\delta(\symbf{x})=x_1$と十分統計量$T(\symbf{x})=\sum_{k=1}^Nx_k$を用いてラオ--ブラックウェルの定理に基づいて、よりよい推定関数$\eta\circ T$をもとめよ。
\end{enumerate}
\end{frame}


\end{document}
