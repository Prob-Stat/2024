\documentclass[lualatex,handout]{beamer}
\setbeamertemplate{footline}[frame number]
\usepackage{luatexja}
\usepackage{amsmath,amssymb}

\usepackage{thm-restate}

%\usetheme{Berlin}
\usecolortheme{rose}

\usepackage{tikz}
\usepackage{pgfplots}
\pgfplotsset{compat=1.18}

%\usepackage[haranoaji]{luatexja-preset}
\usepackage[deluxe,ipaex]{luatexja-preset}
\renewcommand{\kanjifamilydefault}{\gtdefault}
%\setmainjfont{HaranoAjiGothic-Regular}

\usepackage{unicode-math}
%\setmathfont{Fira Math}
\setmathfont{STIX Two Math}
\setmathfont{STIX Two Math}[range=bfup/{Latin,latin,num,Greek,greek}]
\setmathfont{STIX Two Math}[range=bfit/{Latin,latin}]
\setmathfont{STIX Two Math}[range={"0000-"FFFF}]
\setmathrm{STIX Two Math}[StylisticSet=8]

%\usefonttheme{professionalfonts}

\usepackage{luacolor}

\newcommand{\mycolor}[2]{%
  \begingroup
  \colorlet{currentcolor}{.}%
  \color{#1}#2%
  \color{currentcolor}%
  \endgroup
}
\newcommand{\emm}[1]{\mycolor{red}{#1}}
\newcommand{\expt}[1]{\mathbb{E}\left[#1\right]}
\newcommand{\var}[1]{\mathbb{V}\left[#1\right]}
\newcommand{\cov}[1]{\mathsf{Cov}\left[#1\right]}
\newcommand{\vc}[1]{\mathsf{Var}\left[#1\right]}

\DeclareMathOperator*{\argmax}{arg\,max}


\usepackage{xspace}
%\usepackage{bm}
%\newcommand\bm[1]{{\mathbf{#1}}}
\newcommand\defiff{\stackrel{\text{def}}{\iff}}
\newcommand\dx{{\,\mathrm{d}x}}
\newcommand\KL[2]{D\left(#1\,\|\,#2\right)}
\newcommand\dtv{d_{\mathrm{TV}}}

\theoremstyle{definition}
%\newtheorem{fact}{Fact}

\title{確率・統計基礎: 点推定、不偏推定量}
\author{森 立平}
\date{}



\begin{document}
\begin{frame}[plain]
\maketitle
\end{frame}


\begin{frame}{二乗誤差}
\small
\begin{align*}
\mathrm{MSE}(\theta,\,\delta)&\coloneq \expt{(\delta(X)-g(\theta))^2\mid\theta}\\
&= \expt{\left(\delta(X)\emm{-\expt{\delta(X)\mid\theta}+\expt{\delta(X)\mid\theta}}-g(\theta)\right)^2\mid\theta}\\
&= \expt{\left(\delta(X)-\expt{\delta(X)\mid\theta}\right)^2\mid\theta}+\expt{\left(\expt{\delta(X)\mid\theta}-g(\theta)\right)^2\mid\theta}\\
&\quad + 2\expt{\left(\delta(X)-\expt{\delta(X)\mid\theta}\right)\left(\expt{\delta(X)\mid\theta}-g(\theta)\right)\mid\theta}\\
&= \var{\delta(X)\mid\theta}+\left(\expt{\delta(X)\mid\theta}-g(\theta)\right)^2\\
&= \var{\delta(X)\mid\theta}+\mathrm{Bias}_g(\delta\mid\theta)^2
%\expt{\widehat{\theta}(X)^2 - 2\widehat{\theta}(X)\theta + \theta^2}
\end{align*}
\begin{align*}
\mathrm{Bias}_g(\delta\mid\theta)&\coloneq\expt{\delta(X)\mid\theta}-g(\theta)
\end{align*}
\begin{definition}[不偏推定量]
推定量$\delta\colon \mathcal{X}\to\mathbb{R}$が以下を満たすとき、$g(\theta)$の\emm{不偏推定量}であるという。
\begin{align*}
%\mathrm{Bias}_g(\delta\mid\theta) &= 0 \qquad\forall \theta\in\Theta.
\expt{\delta(X)\mid\theta} &= g(\theta)\qquad\forall \theta\in\Theta.
\end{align*}
\end{definition}
\end{frame}

\begin{frame}{標本平均}
\small
$g(\theta) = \expt{X\mid\theta}$とすると、$\delta(X)=X$は不偏推定量。
独立に$N$個サンプルする場合を考える。
\begin{align*}
p(\symbf{x}\mid \theta) &= \prod_{k=1}^N p(x_k\mid \theta).
\end{align*}
\begin{align*}
\delta(\symbf{X}) &= \frac1N\sum_k X_k\qquad\text{(\emm{標本平均})}
\end{align*}
これも不偏推定量。分散は改善する。
\begin{align*}
\var{\frac1N\sum_kX_k\mid\theta}&=
\frac1{N^2}\var{\sum_kX_k\mid\theta}=
\frac1N\var{X\mid\theta}.
\end{align*}
\end{frame}

\begin{frame}{標本分散、不偏分散}
\scriptsize
$g(\theta) = \var{X\mid\theta}$とする。
\begin{align*}
\delta(\symbf{X}) &= \frac1N\sum_k\left(X_k-\frac1N\sum_\ell X_\ell\right)^2\qquad\text{(\emm{標本分散})}
\end{align*}
\begin{align*}
\expt{\delta(\symbf{X})\mid\theta}&=\sum_k\expt{\frac1N\left(X_k-\frac1N\sum_\ell X_\ell\right)^2\mid\theta}\\
&=\frac1N\sum_k\expt{\left(X_k\emm{-\expt{X\mid\theta}+\expt{X\mid\theta}}-\frac1N\sum_\ell X_\ell\right)^2\mid\theta}\\
&=\frac1N\sum_k\expt{\left(X_k-\expt{X\mid\theta}\right)^2\mid\theta}+\expt{\left(\expt{X\mid\theta}-\frac1N\sum_\ell X_\ell\right)^2\mid\theta}\\
&\qquad + \frac2N\sum_k\expt{(X_k-\expt{X\mid\theta})\left(\expt{X\mid\theta}-\frac1N\sum_\ell X_\ell\right)\mid\theta}\\
&=\var{X\mid\theta}-\expt{\left(\expt{X\mid\theta}-\frac1N\sum_\ell X_\ell\right)^2\mid\theta}\\
&=\var{X\mid\theta}-\var{\frac1N\sum_\ell X_\ell\mid\theta}=\emm{\frac{N-1}N}\var{X\mid\theta}.
\end{align*}

\vspace{-1.0em}
\begin{align*}
\delta(\symbf{X}) &= \emm{\frac1{N-1}}\sum_k\left(X_k-\frac1N\sum_\ell X_\ell\right)^2\qquad\text{(\emm{不偏分散})}
\end{align*}
\end{frame}

\begin{frame}{不偏推定量は存在するとは限らない}
パラメータ$\theta\in(0,1)$について、
$X\sim\mathrm{Ber}(\theta)$、$g(\theta) = 1/\theta$とする。
$\delta\colon \{0,1\}\to \mathbb{R}$を不偏推定量とすると、
\begin{align*}
\frac1\theta &= \expt{\delta(X)\mid\theta} = (1-\theta)\delta(0) + \theta \delta(1)
\end{align*}
$\theta\to 0$の極限で左辺は$+\infty$に発散する。一方で右辺は$\delta(0)$に収束する。
よって\emm{$\delta$をどのように定めても不偏推定量にはならない}。
\end{frame}

\if0
\begin{frame}{ジャックナイフサンプリング}
\begin{align*}
\Pr(\symbf{X}=\symbf{x}\mid\theta) &= \prod_{k=1}^N \Pr(X_k=x_k\mid\theta)
\end{align*}
各$N\in\mathbb{N}$について
$\delta_N\colon \mathcal{X}^N\to\mathbb{R}$が
\begin{align*}
\expt{\delta_N(\symbf{X})\mid\theta} &= g(\theta) + O\left(\frac{1}{N}\right)
\end{align*}
を満たすとする。
\end{frame}
\fi

\begin{frame}{不偏推定量の自由度}
\begin{fact}
$\delta_0\colon\mathcal{X}\to\mathbb{R}$を$g(\theta)$の不偏推定量とする。
任意の推定量$\delta\colon\mathcal{X}\to\mathbb{R}$について以下は同値
\begin{enumerate}
\item $\delta$は$g(\theta)$の不偏推定量
\item $\delta-\delta_0$は $0$の不偏推定量
\end{enumerate}
\end{fact}
よって$h\colon\mathcal{X}\to\mathbb{R}$を$0$の不偏推定量とすると、$g(\theta)$の任意の不偏推定量は$\delta=\delta_0+h$と表せる。このとき、
\begin{align*}
\var{\delta(X)\mid\theta}&=
\var{\delta_0(X)+h(X)\mid\theta}\\
&= \var{\delta_0(X)\mid\theta}+\var{h(X)\mid\theta}+2\cov{\delta_0(X), h(X)\mid\theta}\\
&= \var{\delta_0(X)\mid\theta}+\emm{\expt{h(X)^2\mid\theta}+2\expt{\delta_0(X)h(X)\mid\theta}}
%&=
%\expt{(\delta_0(X)+h(X))^2\mid\theta}-
%\expt{\delta_0(X)+h(X)\mid\theta}^2\\
%&=
%\emm{\expt{(\delta_0(X)+h(X))^2\mid\theta}}- g(\theta)^2
\end{align*}
%よって$\emm{\expt{(\delta_0(X)+h(X))^2\mid\theta}}$を最小化する$h$を見つければよい。
よって$\emm{\expt{h(X)^2\mid\theta}+2\expt{\delta_0(X)h(X)\mid\theta}}$を最小化する$h$を見つければよい。
\end{frame}

\begin{frame}{例: 不偏推定量の最適化 1/2}
\small
パラメータ$\theta\in(0,1)$について、
\begin{align*}
\Pr(X=-1\mid \theta) &= \theta\\
\Pr(X=x\mid \theta) &= \theta^x (1-\theta)^2 \qquad\forall x\in\mathbb{Z}_{\ge 0}
\end{align*}
とする(確率$\theta$で$-1$、確率$1-\theta$で幾何分布$\mathrm{Geo}(\theta)$)。このとき、$\emm{\theta}$と$\emm{(1-\theta)^2}$の不偏推定量をそれぞれ
\begin{align*}
\delta_0(x) &=\mathbb{1}_{\{x = -1\}}\\
\delta_1(x) &=\mathbb{1}_{\{x = 0\}}
\end{align*}
とする。
また、$h\colon\mathbb{Z}_{\ge -1}\to\mathbb{R}$を$0$の不偏推定量とすると、
\begin{align*}
&\theta h(-1) + \sum_{x\ge 0} h(x) \theta^x (1-\theta)^2 = 0\qquad\forall \theta\in(0,1)\\
%&h(0) + (h(-1) + h(1) - 2h(0))p + \sum_{x\ge 2} (h(x)-2h(x-1)+h(x-2)) p^x = 0\qquad\forall p\in(0,1)\\
\iff&h(0) + \sum_{x\ge 1} (h(x)-2h(x-1)+h(x-2)) \theta^x = 0\qquad\forall \theta\in(0,1)\\
\iff&h(0)=0,\qquad h(x)-2h(x-1)+h(x-2) = 0\qquad\forall x\in\mathbb{Z}_{\ge1}\\
\iff&\emm{h(x)=xh(1)}\qquad\forall x\in\mathbb{Z}_{\ge-1}\\
\end{align*}
\end{frame}

\begin{frame}{例: 不偏推定量の最適化 2/2}
\small
%\begin{align*}
% \emm{\expt{(\delta_0(X)+h(X))^2\mid p}}
%&= \expt{(\mathbb{1}_{\{X=-1\}}+Xh(1))^2\mid p}\\
%&= \expt{\mathbb{1}_{\{X=-1\}}+2\mathbb{1}_{\{X=-1\}}Xh(1) + X^2h(1)^2\mid p}\\
%&= p -2ph(1) + \left(p + (1-p)\frac{p+p^2}{(1-p)^2}\right)h(1)^2\\
%&= p -2ph(1) + \frac{2p}{1-p}h(1)^2
%\end{align*}
%よって$h(1) = \frac{1-p}2$のとき最小化される。最適な不偏推定量は$p$毎に違う。
\begin{align*}
&\emm{\expt{h(X)^2\mid \theta}+2\expt{\delta_0(X) h(X)\mid \theta}}\\
&=\expt{(\emm{h(1)}X)^2\mid \theta} + 2\expt{\delta_0(X) \emm{h(1)}X\mid \theta}\\
&=\expt{X^2\mid \theta}\emm{h(1)}^2 + 2\expt{\delta_0(X) X\mid \theta}\emm{h(1)}
%&=\expt{(Xh(1))^2\mid\theta} + 2\expt{(\mathbb{1}_{\{X=-1\}}-p)Xh(1)\mid\theta}\\
%&=\expt{X^2\mid\theta}h(1)^2 + 2\expt{(\mathbb{1}_{\{X=-1\}}-p)X\mid\theta}h(1)\\
\end{align*}
よって$\emm{h(1)} = -\frac{\expt{\delta_0(X)X\mid \theta}}{\expt{X^2\mid \theta}}=\frac{1-\theta}2$のとき最小化される。
\emm{最適な不偏推定量は$\theta$毎に違う}。

\vspace{2em}
%\begin{align*}
 %\emm{\expt{(\delta_1(X)+h(X))^2\mid p}}
%&= \expt{(\mathbb{1}_{\{X=0\}}+Xh(1))^2\mid p}\\
%&= \expt{\mathbb{1}_{\{X=0\}}+2\mathbb{1}_{\{X=0\}}Xh(1) + X^2h(1)^2\mid p}\\
%&= (1-p)^2 + \frac{2p}{1-p}h(1)^2
%\end{align*}
%よって$h(1) = 0$のとき最小化される。\emm{最適な不偏推定量は$p$によらず}$\delta_1(x)=\mathbb{1}_{\{x=0\}}$.
同様に$\delta_1$について考えると$\emm{h(1)} = -\frac{\expt{\delta_1(X)X\mid \theta}}{\expt{X^2\mid \theta}}=0$のとき最小化される。
\emm{最適な不偏推定量は$\theta$によらず}$\delta_1(x)=\mathbb{1}_{\{x=0\}}$.
\end{frame}

\begin{frame}{一様最小分散不偏推定量}
\small
\begin{definition}[一様最小分散不偏推定量 (uniform minimum variance unbiased (UMVU) estimator)]
不偏推定量$\delta\colon\mathcal{X}\to\mathbb{R}$が
%任意の$\theta\in\Theta$について、最小の分散を持つとき
以下の条件を満たすとき\emm{一様最小分散不偏推定量}という。
\begin{enumerate}
\item (不偏推定量である) $\expt{\delta(X)\mid\theta}=g(\theta)$ for all $\theta\in\Theta$.
\item (分散が最小である) 
\emm{任意の不偏推定量}$\delta'\colon\mathcal{X}\to\mathbb{R}$と$\theta\in\Theta$について
$\var{\delta(X)\mid\theta}\le \var{\delta'(X)\mid\theta}$が成り立つ。
\end{enumerate}
\end{definition}
%\begin{align*}
%\var{\delta(X)+ah(X)\mid\theta}
%&= \var{\delta(X)\mid\theta}+\var{ah(X)\mid\theta}+2\cov{\delta(X), ah(X)\mid\theta}\\
%&= \var{\delta(X)\mid\theta}+a^2\var{h(X)\mid\theta}+2a\cov{\delta(X), h(X)\mid\theta}
%\end{align*}
%よって$\delta$がUMVU推定量のとき、0の任意の不偏推定量$h$について$\cov{\delta(X), h(X)\mid\theta}=0$でなければならない。
\end{frame}

\begin{frame}{一様最小分散不偏推定量}
\small
\begin{theorem}

\vspace{-2em}
\begin{align*}
\Delta&\coloneq\{\delta\colon\mathcal{X}\to\mathbb{R}\mid \expt{\delta(X)^2\mid\theta}<\infty \quad\forall\theta\in\Theta\}\\
\mathcal{U}&\coloneq\{h\colon\mathcal{X}\to\mathbb{R}\mid \expt{h(X)\mid\theta}=0 \quad\forall\theta\in\Theta\} \cap\Delta
\end{align*}
とする。$\delta\in\Delta$について以下は同値
\begin{enumerate}
\item $\delta$は$\expt{\delta(X)\mid\theta}$のUMVU推定量。
\item 任意の$h\in\mathcal{U}$と$\theta\in\Theta$について$\emm{\expt{\delta(X)h(X)\mid\theta}=0}$。
\end{enumerate}
\end{theorem}
\begin{proof}
$1\implies 2$:
$\delta$は$\expt{\delta(X)\mid\theta}$のUMVU推定量と仮定する。
任意の$h\in\mathcal{U}$と$\theta\in\Theta$と$a\in\mathbb{R}$について
\begin{align*}
\var{\delta(X)\mid\theta}&\le\var{\delta(X)+ah(X)\mid\theta}\\
 &= \var{\delta(X)\mid\theta}+\expt{h(X)^2\mid\theta}a^2+2\expt{\delta(X)h(X)\mid\theta}a
\end{align*}
が成り立つので
($h(x)=0$のときとそうでないときで場合分けすると)
、$\expt{\delta(X)h(X)\mid\theta}=0$。
\end{proof}
\end{frame}

\begin{frame}{一様最小分散不偏推定量}
\small
\begin{theorem}

\vspace{-2em}
\begin{align*}
\Delta&\coloneq\{\delta\colon\mathcal{X}\to\mathbb{R}\mid \expt{\delta(X)^2\mid\theta}<\infty \quad\forall\theta\in\Theta\}\\
\mathcal{U}&\coloneq\{h\colon\mathcal{X}\to\mathbb{R}\mid \expt{h(X)\mid\theta}=0 \quad\forall\theta\in\Theta\} \cap\Delta
\end{align*}
とする。$\delta\in\Delta$について以下は同値
\begin{enumerate}
\item $\delta$は$\expt{\delta(X)\mid\theta}$のUMVU推定量。
\item 任意の$h\in\mathcal{U}$と$\theta\in\Theta$について$\emm{\expt{\delta(X)h(X)\mid\theta}=0}$。
\end{enumerate}
\end{theorem}
\begin{proof}
$2\implies 1$:
任意の$h\in\mathcal{U}$と$\theta\in\Theta$について$\expt{\delta(X)h(X)\mid\theta}=0$ と仮定する。
$\delta'\colon\mathcal{X}\to\mathbb{R}$を$\expt{\delta(X)\mid\theta}$の不偏推定量とする。
$\var{\delta'(X)\mid\theta}=\infty$の場合は
$\var{\delta(X)\mid\theta}\le\var{\delta'(X)\mid\theta}$である。
$\delta'\in\Delta$の場合は$h\coloneq\delta'-\delta\in\mathcal{U}$。
このとき任意の$\theta\in\Theta$について
\begin{align*}
\var{\delta'(X)\mid\theta} &= \var{\delta(X)+h(X)\mid\theta}\\
&= \var{\delta(X)}+\expt{h(X)^2\mid\theta} + 2\expt{\delta(X)h(X)\mid\theta}\\
&= \var{\delta(X)}+\expt{h(X)^2\mid\theta} \ge \var{\delta(X)}
\end{align*}
\end{proof}
\end{frame}

\begin{frame}{推定量の平均}
\footnotesize
\begin{lemma}
$\delta_1,\,\delta_2,\dotsc,\delta_n\colon \mathcal{X}\to\Theta$が$g(\theta)$の不偏推定量のとき、
\begin{align*}
\delta^* &\coloneq \emm{\frac1n\sum_{k=1}^n\delta_k}
\end{align*}
は$g(\theta)$の不偏推定量で
\begin{align*}
\var{\delta^*(X)\mid\theta} &\le\frac1n\sum_{k=1}^n\var{\delta_k(X)\mid\theta}
\qquad\forall\theta\in\Theta.
\end{align*}
\end{lemma}
\begin{proof}
$\delta^*$が不偏推定量であることは期待値の線形性から明らか。
\begin{align*}
\var{\delta^*(X)\mid\theta} &=
\expt{\left(\frac1n\sum_{k=1}^n\delta_k(X) - g(\theta)\right)^2\mid\theta}\\
&\le
\frac1n\sum_{k=1}^n\expt{\left(\delta_k(X) - g(\theta)\right)^2\mid\theta}\quad\text{(イェンセンの不等式)}\qedhere
%\frac1{n^2} \left(\sum_{k=1}^n \var{\widehat{\theta}_k(X)\mid\theta} + 2\sum_{k<\ell} \cov{\widehat{\theta}_k,\,\widehat{\theta}_\ell}\right)
\end{align*}
\end{proof}
\end{frame}

\begin{frame}{対称な不偏推定量}
\small
\begin{align*}
p(\symbf{x}\mid \theta) &= \prod_{k=1}^N p(x_k\mid \theta).
\end{align*}
\begin{lemma}[対称な不偏推定量]
任意の不偏推定量$\delta\colon \mathcal{X}^N\to\Theta$について、ある($x_1,\dotsc,x_n$について)\emm{対称な}不偏推定量$\delta^*$が存在して
\begin{align*}
\var{\delta^*(\symbf{X})\mid\theta} &\le \var{\delta(\symbf{X})\mid\theta}
\qquad\forall\theta\in\Theta.
\end{align*}
\end{lemma}
\begin{proof}
\begin{align*}
\delta^*(x_1,\dotsc,x_N) &\coloneq \frac1{N!} \sum_{\sigma\in S_N} \delta\left(x_{\sigma(1)},\dotsc,x_{\sigma(N)}\right)
\end{align*}
と定義すればよい。
\end{proof}
\end{frame}

\begin{frame}{コーシー--シュワルツの不等式}
\small
\begin{lemma}
任意の確率変数$X$と$Y$について
\begin{align*}
|\expt{XY}|&\le\sqrt{\expt{X^2}\expt{Y^2}}.
\end{align*}
\end{lemma}
\begin{proof}
任意の$a\in\mathbb{R}$について
\begin{align*}
0&\le\expt{(X+aY)^2}\\
&\le\expt{X^2}+\expt{Y^2}a^2+2\expt{XY}a
\end{align*}
$\expt{Y^2}=0$のとき、$\Pr(Y=0)=1$で不等式は成り立つ。
$\expt{Y^2}>0$とする。
上の式に$a=-\frac{\expt{XY}}{\expt{Y^2}}$を代入すると、
\begin{align*}
0&\le \expt{X^2} - \frac{\expt{XY}^2}{\expt{Y^2}}
\end{align*}
よって不等式は得られた。
\end{proof}
\end{frame}

\begin{frame}{共分散の不等式}
\small
\begin{corollary}
任意の確率変数$X$と$Y$について
\begin{align*}
|\cov{X,\,Y}|&\le\sqrt{\var{X}\var{Y}}.
\end{align*}
\end{corollary}
\begin{proof}
$A= X - \expt{X}$,
$B= Y - \expt{Y}$について、コーシー--シュワルツの不等式より
\begin{align*}
&|\expt{AB}|\le\sqrt{\expt{A^2}\expt{B^2}}\\
\iff&|\cov{X,\,Y}|\le\sqrt{\var{X}\var{Y}}.
\end{align*}
\end{proof}
\end{frame}

\begin{frame}{不偏推定量の分散の下界}
不偏推定量$\delta\colon\mathcal{X}\to\mathbb{R}$の分散の下界を導出したい。

\vspace{1em}
任意の$\emm{\psi\colon\mathcal{X}\times\Theta\to\mathbb{R}}$について
\begin{align*}
\var{\delta(X)\mid\theta}&\ge\frac{\cov{\delta(X),\,\psi(X,\,\theta)\mid\theta}^2}{\var{\psi(X,\,\theta)\mid\theta}}.
\end{align*}
ここで$p(x\mid\theta)$が\emm{$\theta$で微分可能}なとき
\begin{align*}
\psi(x,\,\theta)&
=\emm{\frac{\partial\log p(x\mid\theta)}{\partial\theta}}
=\frac1{p(x\mid\theta)}\frac{\partial p(x\mid\theta)}{\partial\theta}
\end{align*}
とおく(離散の場合は$p(x\mid\theta)$は\emm{確率質量関数}としての尤度関数に置き換える)と、
\begin{align*}
\expt{\psi(X,\,\theta)\mid\theta}&= \int \frac{\partial p(x\mid\theta)}{\partial\theta}\mathrm{d}x\\
&= \frac{\partial }{\partial\theta}\int p(x\mid\theta)\mathrm{d}x=0.\qquad\text{(微分と積分の交換)}
\end{align*}
\end{frame}

\begin{frame}{フィッシャー情報量}
\small
\begin{definition}[フィッシャー情報量]
\vspace{-1em}
\begin{align*}
I(\theta) &\coloneq \emm{\expt{\left(\frac{\partial \log p(X\mid\theta)}{\partial\theta}\right)^2\mid\theta}}.
%&= -\expt{\frac{\partial^2}{\partial\theta^2}\log p(X\mid\theta)\mid\theta}.
%&= \expt{\left(\frac{1}{p(X\mid\theta)}\frac{\partial p(X\mid\theta)}{\partial\theta}\right)^2\mid\theta}\\
%&= \int p(x\mid\theta)\left(\frac{1}{p(x\mid\theta)}\frac{\partial p(x\mid\theta)}{\partial\theta}\right)^2\mathrm{d}x\\
%&= \int \frac{1}{p(x\mid\theta)}\left(\frac{\partial p(x\mid\theta)}{\partial\theta}\right)^2\mathrm{d}x
\end{align*}
\end{definition}
\begin{align*}
\var{\delta(X)\mid\theta}&\ge\frac{\cov{\delta(X),\,\psi(X,\,\theta)\mid\theta}^2}{\var{\psi(X,\,\theta)\mid\theta}}\\
&=\frac{\expt{\delta(X)\psi(X,\,\theta)\mid\theta}^2}{\emm{I(\theta)}}.
\end{align*}
\begin{align*}
\expt{\delta(X)\psi(X,\,\theta)\mid\theta}&=
\int p(x\mid\theta) \delta(x)\frac1{p(x\mid\theta)}\frac{\partial p(x\mid\theta)}{\partial\theta}\mathrm{d}x\\
&=
\frac{\partial}{\partial\theta}\int p(x\mid\theta) \delta(x)\mathrm{d}x\qquad\text{(微分と積分の交換)}\\
&=
\frac{\partial}{\partial\theta}g(\theta)=g'(\theta).
\end{align*}
\end{frame}

\begin{frame}{クラメール--ラオの不等式}
\begin{theorem}[クラメール--ラオの不等式]
$\delta\colon\mathcal{X}\to\mathbb{R}$を$g(\theta)$の任意の不偏推定量とするとき以下が成り立つ。
\begin{align*}
\var{\delta(X)\mid\theta} &\ge \emm{\frac{g'(\theta)^2}{I(\theta)}}.
\end{align*}
\end{theorem}
\end{frame}

\if0
\begin{frame}{加法的ノイズのモデル}
\footnotesize
\begin{align*}
X &= \theta + Z
\end{align*}
\begin{itemize}
\item $\expt{Z} = 0$.
\item $\var{Z} < \infty$.
\end{itemize}
\begin{theorem}
$\widehat{\theta}^*$が$Z$の分布によらず対称な不偏推定量となるとき、$\widehat{\theta}^*$は\emm{標本平均}
\begin{align*}
\widehat{\theta}^*(x_1,\dotsc,x_n)&\coloneq \frac1n \sum_{k=1}^n x_k
\end{align*}
である。
\end{theorem}
\begin{proof}
$n=1$のとき、
$\Pr(Z=0)=1$と仮定すると、
\begin{align*}
\theta&= \expt{\widehat{\theta}^*(X)\mid\theta}
= \widehat{\theta}^*(\theta)
\end{align*}
が任意の$\theta\in\Theta$について成り立つ。よって$\widehat{\theta}^*(x)=x$である。
\end{proof}
\end{frame}

\begin{frame}{加法的ノイズのモデル}
\small
ある$a\ne 0$と$p\in[0,1]$について、ノイズの分布を
\begin{align*}
\Pr(Z = a(p-1)) &= p&
\Pr(Z = ap) &= 1-p
\end{align*}
とすると$\expt{Z}=0$, $\var{Z}<\infty$である。

$n=2$のとき、
任意の$a\ne 0$と$p\in[0,1]$について
\begin{align*}
%\expt{\widehat{\theta}^*(\symbf{X})\mid\theta} &=
\theta&= \expt{\widehat{\theta}^*(X_1,X_2)\mid\theta}\\
 &= p^2 \widehat{\theta}^*(a(p-1),a(p-1)) + (1-p)^2 \widehat{\theta}^*(ap, ap)\\
&\quad + p(1-p) \widehat{\theta}^*(a(p-1), ap) + (1-p)p\widehat{\theta}^*(ap, a(p-1))\\
 &= p^2 a(p-1) + (1-p)^2 ap + 2p(1-p) \widehat{\theta}^*(ap, a(p-1))
\end{align*}
が成り立つ。
\end{frame}
\fi

\begin{frame}{課題}
\small
パラメータ$\theta\in(0,1)$について、$X_1,\dotsc,X_N$が独立なベルヌーイ分布$\mathrm{Ber}(\theta)$に従うものとする。
つまり、以下が成り立つ。
\begin{align*}
\Pr(X=x_1,\dotsc,X_N=x_N\mid\theta)&=\theta^{x_1+\dotsb+x_N}(1-\theta)^{N-(x_1+\dotsb+x_N)}.
\end{align*}
$\symbf{X}$から$g(\theta)$を推定する問題を考える。
$T(\symbf{x})=\frac1N\sum_kx_k$とする。
以下の問に答えよ。

\vspace{1em}
\begin{enumerate}
\setlength{\itemsep}{2em}
\item $\symbf{X}$のフィッシャー情報量$I(\theta)$をもとめよ。
\item $g(\theta)=\expt{X\mid\theta}=\theta$の不偏推定量$\delta(\symbf{x})=T(\symbf{x})$の分散とクラメール--ラオの下界を導出せよ。
\item $g(\theta)=\var{X\mid\theta}=\theta(1-\theta)$の不偏推定量である不偏分散$\delta(\symbf{x})=\frac1{N-1}\sum_k(x_k-T(\symbf{x}))^2$を$T(\symbf{x})$の関数として表せ。また、$g(\theta)$の不偏推定量の分散に対するクラメール--ラオの下界を導出せよ。
\end{enumerate}
\end{frame}


\end{document}
