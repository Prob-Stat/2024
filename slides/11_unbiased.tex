\documentclass[lualatex,handout]{beamer}
\setbeamertemplate{footline}[frame number]
\usepackage{luatexja}
\usepackage{amsmath,amssymb}

\usepackage{thm-restate}

%\usetheme{Berlin}
\usecolortheme{rose}

\usepackage{tikz}
\usepackage{pgfplots}
\pgfplotsset{compat=1.18}

%\usepackage[haranoaji]{luatexja-preset}
\usepackage[deluxe,ipaex]{luatexja-preset}
\renewcommand{\kanjifamilydefault}{\gtdefault}
%\setmainjfont{HaranoAjiGothic-Regular}

\usepackage{unicode-math}
%\setmathfont{Fira Math}
\setmathfont{STIX Two Math}
\setmathfont{STIX Two Math}[range=bfup/{Latin,latin,num,Greek,greek}]
\setmathfont{STIX Two Math}[range=bfit/{Latin,latin}]
\setmathfont{STIX Two Math}[range={"0000-"FFFF}]
\setmathrm{STIX Two Math}[StylisticSet=8]

%\usefonttheme{professionalfonts}

\usepackage{luacolor}

\newcommand{\mycolor}[2]{%
  \begingroup
  \colorlet{currentcolor}{.}%
  \color{#1}#2%
  \color{currentcolor}%
  \endgroup
}
\newcommand{\emm}[1]{\mycolor{red}{#1}}
\newcommand{\expt}[1]{\mathbb{E}\left[#1\right]}
\newcommand{\var}[1]{\mathbb{V}\left[#1\right]}
\newcommand{\cov}[1]{\mathsf{Cov}\left[#1\right]}
\newcommand{\vc}[1]{\mathsf{Var}\left[#1\right]}

\DeclareMathOperator*{\argmax}{arg\,max}


\usepackage{xspace}
%\usepackage{bm}
%\newcommand\bm[1]{{\mathbf{#1}}}
\newcommand\defiff{\stackrel{\text{def}}{\iff}}
\newcommand\dx{{\,\mathrm{d}x}}
\newcommand\KL[2]{D\left(#1\,\|\,#2\right)}
\newcommand\dtv{d_{\mathrm{TV}}}

\theoremstyle{definition}

\title{確率・統計基礎: 点推定、不偏推定量}
\author{森 立平}
\date{}



\begin{document}
\begin{frame}[plain]
\maketitle
\end{frame}


\begin{frame}{二乗誤差}
\small
\begin{align*}
\mathrm{MSE}(\theta,\,\delta)&\coloneq \expt{(\delta(X)-g(\theta))^2\mid\theta}\\
&= \expt{\left(\delta(X)\emm{-\expt{\delta(X)\mid\theta}+\expt{\delta(X)\mid\theta}}-g(\theta)\right)^2\mid\theta}\\
&= \expt{\left(\delta(X)-\expt{\delta(X)\mid\theta}\right)^2\mid\theta}+\expt{\left(\expt{\delta(X)\mid\theta}-g(\theta)\right)^2\mid\theta}\\
&\quad + 2\expt{\left(\delta(X)-\expt{\delta(X)\mid\theta}\right)\left(\expt{\delta(X)\mid\theta}-g(\theta)\right)\mid\theta}\\
&= \var{\delta(X)\mid\theta}+\left(\expt{\delta(X)\mid\theta}-g(\theta)\right)^2\\
&= \var{\delta(X)\mid\theta}+\mathrm{Bias}_g(\delta\mid\theta)^2
%\expt{\widehat{\theta}(X)^2 - 2\widehat{\theta}(X)\theta + \theta^2}
\end{align*}
\begin{align*}
\mathrm{Bias}_g(\delta\mid\theta)&\coloneq\expt{\delta(X)\mid\theta}-g(\theta)
\end{align*}
\begin{definition}[不偏推定量]
推定量$\delta\colon \mathcal{X}\to\mathbb{R}$が以下を満たすとき、$g(\theta)$の\emm{不偏推定量}であるという。
\begin{align*}
\mathrm{Bias}_g(\delta\mid\theta) &= 0 \qquad\forall \theta\in\Theta.
\end{align*}
\end{definition}
\end{frame}

\begin{frame}{不偏推定量は存在するとは限らない}

\end{frame}

\begin{frame}{ジャックナイフサンプリング}

\end{frame}

\begin{frame}{推定量の平均}
\footnotesize
\begin{lemma}
$\widehat{\theta}_1,\,\widehat{\theta}_2,\dotsc,\widehat{\theta}_n\colon A\to\Theta$が不偏推定量のとき、
\begin{align*}
\widehat{\theta}^* &\coloneq \frac1n\sum_{k=1}^n\widehat{\theta}_k
\end{align*}
は不偏推定量で
\begin{align*}
\var{\widehat{\theta}^*(X)\mid\theta} &\le\frac1n\sum_{k=1}^n\var{\widehat{\theta}_k(X)\mid\theta}
\qquad\forall\theta\in\Theta.
\end{align*}
\end{lemma}
\begin{proof}
$\widehat{\theta}^*$が不偏推定量であることは期待値の線形性から明らか。
\begin{align*}
\var{\widehat{\theta}^*(X)\mid\theta} &=
\expt{\left(\frac1n\sum_{k=1}^n\widehat{\theta}_k(X) - \theta\right)^2\mid\theta}\\
&\le
\frac1n\sum_{k=1}^n\expt{\left(\widehat{\theta}_k(X) - \theta\right)^2\mid\theta}\quad\text{(イェンセンの不等式)}\qedhere
%\frac1{n^2} \left(\sum_{k=1}^n \var{\widehat{\theta}_k(X)\mid\theta} + 2\sum_{k<\ell} \cov{\widehat{\theta}_k,\,\widehat{\theta}_\ell}\right)
\end{align*}
\end{proof}
\end{frame}

\begin{frame}{対称な不偏推定量}
\small
\begin{align*}
p(\symbf{x}\mid \theta) &= \prod_{k=1}^N p(x_k\mid \theta).
\end{align*}
\begin{lemma}[対称な不偏推定量]
任意の不偏推定量$\widehat{\theta}\colon A^N\to\Theta$について、ある対称な不偏推定量$\widehat{\theta}^*$が存在して
\begin{align*}
\var{\widehat{\theta}^*(\symbf{X})\mid\theta} &\le \var{\widehat{\theta}(\symbf{X})\mid\theta}
\qquad\forall\theta\in\Theta.
\end{align*}
\end{lemma}
\begin{proof}
\begin{align*}
\widehat{\theta}^*(x_1,\dotsc,x_N) &\coloneq \frac1{N!} \sum_{\sigma\in S_N} \widehat{\theta}\left(x_{\sigma(1)},\dotsc,x_{\sigma(N)}\right)
\end{align*}
と定義すればよい。
\end{proof}
\end{frame}

\begin{frame}{加法的ノイズのモデル}
\footnotesize
\begin{align*}
X &= \theta + Z
\end{align*}
\begin{itemize}
\item $\expt{Z} = 0$.
\item $\var{Z} < \infty$.
\end{itemize}
\begin{theorem}
$\widehat{\theta}^*$が$Z$の分布によらず対称な不偏推定量となるとき、$\widehat{\theta}^*$は\emm{標本平均}
\begin{align*}
\widehat{\theta}^*(x_1,\dotsc,x_n)&\coloneq \frac1n \sum_{k=1}^n x_k
\end{align*}
である。
\end{theorem}
\begin{proof}
$n=1$のとき、
$\Pr(Z=0)=1$と仮定すると、
\begin{align*}
\theta&= \expt{\widehat{\theta}^*(X)\mid\theta}
= \widehat{\theta}^*(\theta)
\end{align*}
が任意の$\theta\in\Theta$について成り立つ。よって$\widehat{\theta}^*(x)=x$である。
\end{proof}
\end{frame}

\begin{frame}{加法的ノイズのモデル}
\small
ある$a\ne 0$と$p\in[0,1]$について、ノイズの分布を
\begin{align*}
\Pr(Z = a(p-1)) &= p&
\Pr(Z = ap) &= 1-p
\end{align*}
とすると$\expt{Z}=0$, $\var{Z}<\infty$である。

$n=2$のとき、
任意の$a\ne 0$と$p\in[0,1]$について
\begin{align*}
%\expt{\widehat{\theta}^*(\symbf{X})\mid\theta} &=
\theta&= \expt{\widehat{\theta}^*(X_1,X_2)\mid\theta}\\
 &= p^2 \widehat{\theta}^*(a(p-1),a(p-1)) + (1-p)^2 \widehat{\theta}^*(ap, ap)\\
&\quad + p(1-p) \widehat{\theta}^*(a(p-1), ap) + (1-p)p\widehat{\theta}^*(ap, a(p-1))\\
 &= p^2 a(p-1) + (1-p)^2 ap + 2p(1-p) \widehat{\theta}^*(ap, a(p-1))
\end{align*}
が成り立つ。
\end{frame}

\begin{frame}{課題}
\small
以下の問に答えよ。
\begin{enumerate}
\setlength{\itemsep}{1em}
\item 
\begin{align*}
p(\symbf{x}\mid\theta)&= \frac1{\sqrt{(2\pi)^N}}\mathrm{e}^{-\sum_{k=1}^N\frac{(x_k-\theta)^2}2}
\end{align*}
のとき、以下の$T\colon\mathbb{R}^N\to\mathbb{R}$が十分統計量かどうか示せ。結論だけでなく理由も示すこと。
\begin{align*}
T(\symbf{x})&= \sum_{k=1}^N x_k.
\end{align*}
\item
\begin{align*}
\Pr(X_1=x_1,\dotsc,X_N=x_N\mid\lambda)&= \frac1{\prod_{k=1}^N x_k!}\lambda^{\sum_{k=1}^N x_k} \mathrm{e}^{-N\lambda}
\end{align*}
のとき、$\lambda$の推定関数$\delta(\symbf{x})=x_1$と十分統計量$T(\symbf{x})=\sum_{k=1}^Nx_k$を用いてラオ--ブラックウェルの定理に基づいて、よりよい推定関数$\eta\circ T$をもとめよ。
\end{enumerate}
\end{frame}


\end{document}
